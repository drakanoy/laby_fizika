\documentclass[12pt,letterpaper]{article}
\usepackage[left=2cm, right=2cm,top=2cm,bottom=2cm,bindingoffset=0cm]{geometry}
\usepackage[utf8]{inputenc}
\usepackage{amsmath,ragged2e}
\usepackage{multirow}

%Russian-specific packages
%--------------------------------------
\usepackage[T2A]{fontenc}
\usepackage[utf8]{inputenc}
\usepackage[russian]{babel}
\usepackage{amsmath}
\usepackage{tabularx}
\usepackage{array}
\usepackage{graphicx}
\graphicspath{ {./images/} }
% \usepackage{fancyhdr}
%--------------------------------------

%Hyphenation rules
%--------------------------------------
\usepackage{hyphenat}
\hyphenation{ма-те-ма-ти-ка вос-ста-нав-ли-вать}
%--------------------------------------


 \title{ \textbf{Рабочий протокол и отчет по
лабораторной работе № 4}\\ \vspace{1mm}
\begin{flushleft}
{\large
Группа: 101.2\\
Студенты: Сосновицкая Злата, Ясинский Михаил, Чатулов Антон\\
Преподаватель: Павлов Александр Валерьевич\\
Отчет принят:}\\
\end{flushleft}
\author{}
\date{}
}

 \begin{document}
 \maketitle
%  \begin{description}
      \section{Цель работы.}\
      \par Изучить поведение примесного полупроводника в магнитном
поле.
      \par

      \section{Задачи, решаемые при выполнении работы.}\par
      \begin{enumerate}
          \item Провести измерение продольного напряжения $U_{||}$ в зависимости от силы тока через образец Германия (Ge-p).
          \item Построить график $I=I(U_{||})$
          \item По графику определить величину проводимости $\sigma$, используя формулу (1)
          \item Провести измерения поперечного напряжения, изменяя магнитное поле, в которое помещён образец.
          \item Построить график $I=U_{\bot}(B)$
          \item Используя график и формулу (2) определить величину постоянной Холла.
          \item По формуле (3) найти величину концентрации носителей $n$.
          \item По формуле (4) найти величину подвижности заряда $\mu$.

      \end{enumerate}

      \section{Объект исследования.}\


      \par Объект исследования - проводники в электромагнитном поле.\
      \par Предмет исследования - эффект Холла при взаимодействии полупроводника с электромагнитным полем.\par

      \section{Метод экспериментального исследования.}\
      \par Прямые измерения опытным путём — измерение продольного напряжения в зависимости от силы тока, измерения поперечного напряжения от величины магнитного поля;\
      \par Аналитические выводы, косвенное измерение величин для установления экспериментальной зависимости — построение графиков зависимостей физических величин, линейная аппроксимация.


      \section{Рабочие формулы и исходные данные.}\par
        \subsection{Формулы:}
        \begin{enumerate}
            \item Проводимость
                \begin{equation}
                    \sigma  = \frac{\Delta I \cdot l}{\Delta U \cdot S}
                \end{equation}
            \item Постоянная Холла
                \begin{equation}
                    R=\frac{\Delta U_{\bot}\cdot d}{\Delta B\cdot I}
                \end{equation}
            \item Концентрация носителей
                \begin{equation}
                    n=\frac{1}{eR}
                \end{equation}
            \item Подвижность
                \begin{equation}
                    \mu=\sigma R
                \end{equation}
        \end{enumerate}
        \subsection{Исходные данные:}
        \begin{enumerate}
            \item Геометрические параметры образца:
            $$d=1\textit{мм}$$
            $$S=10\textit{мм²}$$
            $$l=20\textit{мм}$$
        \end{enumerate}


      \section{Измерительные приборы.}\par
      \begin{table}[h!]
          \centering
          \caption{\centering{Приборы и их погрешности}}
          \begin{tabular}{|c|c|}
                \hline
               Прибор&Погрешности\\
               \hline
               Амперметр&0,001\textit{А}\\
               \hline
               Вольтметр(||)&0,001\textit{В}\\
               \hline
               Вольтметр(-)&1\cdot10^{-4}\textit{В}\\
               \hline
               Амперметр(магн)&0,1\textit{А}\\
               \hline
               Магнитометр&0,001\textit{Тл}\\
               \hline
          \end{tabular}
          \end{table}


      \section{Результаты прямых измерений и их обработки.}\par

      \begin{table}[h!]
          \centering
          \caption{\centering{Прямые измерения продольного напряжения}}
          \begin{tabular}{|c|c|c|}
                \hline
               № Опыта&I,A&U_{||},B \\
               \hline
               1&0,005&0,211\\
               \hline
               2&0,01&0,455\\
               \hline
               3&0,015&0,709\\
               \hline
               4&0,02&0,984\\
               \hline
               5&0,025&1,216\\
               \hline
               6&0,03&1,475\\
               \hline
          \end{tabular}
          \label{tab:my_label}
      \end{table}

      \begin{table}[h!]
          \centering
          \caption{\centering{Прямые измерения поперечного напряжения}}
          \begin{tabular}{|c|c|c|c|}
            \hline
            № Опыта&I_{mg},A&B,\textit{Тл}&U_{\bot},\textit{мВ}\\
            \hline
            1&0,2&0,012&40,7\\
            \hline
            2&0,4&0,036&44,2\\
            \hline
            3&0,6&0,058&47,5\\
            \hline
            4&0,8&0,086&51,5\\
            \hline
            5&1,0&0,111&55,1\\
            \hline
            6&1,2&0,136&58,7\\
            \hline
            7&1,4&0,158&61,7\\
            \hline
            8&1,6&0,189&66\\
            \hline
            9&1,8&0,222&70,3\\
            \hline
            10&2&0,241&72,7\\
            \hline


          \end{tabular}
          \label{tab:my_label}
      \end{table}
\newpage
      \section{Расчет результатов косвенных измерений.}\par
      \subsection{Проводимость}\par

        \begin{figure}[h!]
            \center{\includegraphics[scale=0.5]{graphs/graph1.png}}
            \caption{Зависимость продольного напряжения от силы тока через образец}
            \label{fig:image}
        \end{figure}

            Величина $\frac{\Delta I}{\Delta U}$ равна тангенсу угла наклона графика (*).$\frac{\Delta I}{\Delta U} = 50,7 \frac{A}{B}$.\newline
            $\sigma  = \frac{\Delta I \cdot l}{\Delta U \cdot S}=50,7\frac{\textit{А}}{\textit{В}}\cdot\frac{20\textit{м}\cdot10^{-3}}{10\textit{м²}\cdot10^{-6}}=101463 \frac{\textit{А}}{\textit{В×м}}$
      \subsection{Постоянная Холла}\par

        \begin{figure}[h!]
            \center{\includegraphics[scale=0.5]{graphs/graph2.png}}
            \caption{Зависимость поперечного напряжения от магнитной индукции}
            \label{fig:image}
        \end{figure}


         $\frac{\Delta U_{\bot}}{\Delta B}=0,14 \frac{\textit{В}}{\textit{Тл}}$(определяется наклоном графика (Рис.2)
         \newline
         $R=\frac{\Delta U_{\bot}\cdot d}{\Delta B\cdot I}=0,14\frac{\textit{В}}{\textit{Тл}}\cdot\frac{1\textit{м}\cdot 10^{-3}}{0,02\textit{A}} = 7,0\cdot10^{-3} \frac{\textit{м³}}{\textit{Кл}}$
      \newpage
      \subsection{Концентрация носителей заряда}\par
        $n=\frac{1}{eR}=\frac{1}{1,6\cdot 10^{-19}\textit{Кл}\cdot 7,0\cdot10^{-3}\frac{\textit{м³}}{\textit{Кл}}}=8,9\cdot 10^{20} \frac{1}{\textit{м³}}$
      \subsection{Подвижность носителей заряда}\newline
        $\mu = \sigma R=101460 \frac{\textit{А}}{\textit{В×м}}\cdot 7,0\cdot10^{-3}\frac{\textit{м³}}{\textit{Кл}}=711\frac{\textit{м²}}{\textit{В×c}}$
      \subsection {Расчёт погрешностей}
        \begin{enumerate}
            \item Погрешность концентрации носителей\par
            $\Delta n = n \frac {\Delta R}{R}=1,6\cdot 10^{20} \frac{1}{\textit{м³}}$
            \item Погрешность проводимости\par
            $\Delta \sigma = 30 \frac{A}{B\cdot \textit{м}}$
            \item Погрешность коэффициента Холла\par
            $\Delta R = 1\cdot10^{-3} \frac{\textit{м³}}{\textit{Кл}}$
            \item Погрешность подвижности носителей заряда\par
            $\Delta \mu = 30 \frac{\textit{м²}}{\textit{В} \cdot \textit {c}} $
        \end{enumerate}
      \section{Выводы и анализ результатов}\
      \begin{table}[h!]
          \centering
          \begin{tabular}{|c|c|}
          \hline
               Проводимость & 101 463 \pm 30 \frac{A}{B\cdot \textit{м}}\\
               \hline
               Постоянная Холла& 70\cdot10^{-2} \pm 10\cdot10^{-2} \frac{\textit{м}^3}{\textit{Кл}}\\
               \hline
               Концентрация носителей заряда&8,9\cdot  10^{20} \pm 1,6\cdot 10^{20}\frac{\textit{1}}{\textit{м³}}\\
               \hline
               Подвижность носителей заряда&711 \pm 30\frac{\textit{м²}}{\textit{В}\cdot\textit{с}}\\
               \hline
          \end{tabular}
          \caption{Конечные результаты}
          \label{tab:my_label}
      \end{table}
      \par Лабораторная работа состояла из нескольких этапов. В первую очередь, посредством прямых измерений продольного напряжения на германиевом проводнике при прохождении через него тока различной силы, была получена вольт-амперная характеристика эксперементальной установки. Таким образом, опытном путём была подтверждена линейная зависимость силы тока от продольного напряжения, в полном соответствии с теоретическими предсказаниями. Эти измерения также позволили определить и проводимость германиевого полупроводника, посредством косвенных вычислений.


      Затем было установлено фиксированное значение силы тока в полупроводнике в 20мА, и полупроводник был помещён во внешнее магнитное поле. После чего, за счёт изменения силы тока в магните, удалось измерить значения поперечного напряжения в полупроводнике при различных показателях магнитной индукции внешнего поля. Зависимость поперечного напряжения от магнитной индукции на практике оказалась линейной, что также целиком соответсвует теоретическим выкладкам. Полученная зависимость была применена в косвенных вычислениях для определения Постоянной Холла, концентрации носителей заряда и его подвижности.


      В ходе расчётов были учтены все необходимые погрешности, благодаря чему удалось получить наиболее точные значения всех необходимых величин.
      \section{Замечания преподавателя}\\
%  \end{description}

 \end{document}