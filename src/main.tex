\documentclass{article}
\usepackage[utf8]{inputenc}
\usepackage[russian]{babel}
\usepackage{amsmath}
\usepackage{amssymb}
\usepackage{textcomp}

\begin{document}
\textbf{Условие задачи:} Пусть случайная величина \( X \) равномерно распределена на отрезке \([-1,1]\), то есть
\[
   f_X(x) \;=\;
   \begin{cases}
      \tfrac12, & x \in [-1,1],\\[4pt]
      0, & \text{иначе}.
   \end{cases}
\]

Требуется найти распределение двумерного случайного вектора
\[
   Y \;=\; \bigl( X^3,\;\lvert 0.5 - X\rvert \bigr).
\]

\textbf{1. Поддержка (support) распределения.}

Очевидно, что \( Y \) принимает значения на кривой, заданной параметрически:
\[
   y_1 \;=\; x^3,
   \quad
   y_2 \;=\; \lvert 0.5 - x \rvert,
   \quad
   x \in [-1,1].
\]
Иными словами,
\[
   \mathrm{Supp}(Y) \;=\; \bigl\{ (y_1,y_2):\, y_1 = x^3,\; y_2 = |0.5 - x|,\; x \in [-1,1] \bigr\}.
\]

\textbf{2. Нахождение плотности вдоль кривой.}

Перейдём к дифференциалу длины дуги. Для параметризации
\(\gamma(x) = \bigl(x^3,\,|0.5 - x|\bigr)\), \(x \in [-1,1]\), длина дуги \(ds\) определяется формулой
\[
   ds \;=\; \sqrt{\Bigl(\frac{d}{dx}x^3\Bigr)^2 +
   \Bigl(\frac{d}{dx}|0.5 - x|\Bigr)^2}\,\mathrm{d}x
   \;=\; \sqrt{(3x^2)^2 + 1}\,\mathrm{d}x
   \;=\; \sqrt{9x^4 + 1}\,\mathrm{d}x.
\]

Так как \( X \) равномерна на \([-1,1]\), вероятность попасть в малый интервал \(\mathrm{d}x\) равна \(\tfrac12\,\mathrm{d}x\). Следовательно, плотность вдоль кривой (в терминах длины дуги) будет:
\[
   f_Y(x) \;=\;
   \frac{ f_X(x) }{ \frac{ds}{dx} }
   \;=\;
   \frac{ \tfrac12 }{ \sqrt{9x^4 + 1} }.
\]

\textbf{3. Итоговое выражение для плотности.}

В терминах координат \((y_1,y_2)\) это означает:
\[
   f_Y(y_1,y_2) \;=\;
   \begin{cases}
     \displaystyle
     \frac{1}{2\,\sqrt{9x^4 + 1}},
     & \text{если }(y_1,y_2) = \bigl(x^3,\;|0.5 - x|\bigr)\text{ для } x\in[-1,1],\\[6pt]
     0, & \text{иначе}.
   \end{cases}
\]

Таким образом, распределение случайного вектора \(\bigl(X^3,\;|0.5 - X|\bigr)\) сосредоточено на кривой \(\{(x^3,\,|0.5 - x|) : x\in[-1,1]\}\) с приведённой выше плотностью относительно меры длины на этой кривой.

\end{document}