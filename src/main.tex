\documentclass{article}
\usepackage[utf8]{inputenc}
\usepackage[russian]{babel}
\usepackage{amsmath}
\usepackage{amssymb}
\usepackage{textcomp}

\begin{document}
\section*{Решение задачи}

Пусть в группе 14 человек, каждый с уникальным днём рождения (все 365 дней равновероятны). Определим случайную величину:
\[
X = \text{число месяцев, в которых есть хотя бы один день рождения}.
\]
Требуется найти математическое ожидание \(\mathbb{E}[X]\) и дисперсию \(\mathrm{D}[X]\).

\subsection*{Математическое ожидание \(\mathbb{E}[X]\)}

Обозначим индикатор события «в месяце \(m\) есть хотя бы один ДР» как \(I_m\). Тогда
\[
X = I_1 + I_2 + \dots + I_{12},
\quad
\mathbb{E}[X] = \sum_{m=1}^{12} \mathbb{E}[I_m].
\]
Заметим, что
\(
\mathbb{E}[I_m] = P(\text{в месяце \(m\) есть хотя бы один ДР}).
\)
Обозначим \(n_m\) за число дней в \(m\)-м месяце. Тогда вероятность, что \textbf{ни} один из 14 разных дней рождений не попадает в месяц \(m\), равна
\[
\frac{C_{365 - n_m}^{14}}{C_{365}^{14}}.
\]
Следовательно,
\[
\mathbb{E}[I_m] = 1 \;-\; \frac{C_{365 - n_m}^{14}}{C_{365}^{14}}.
\]
Таким образом,
\[
\mathbb{E}[X]
= \sum_{m=1}^{12}
\left[
1 - \frac{C_{365 - n_m}^{14}}{C_{365}^{14}}
\right].
\]

Для упрощения делим месяцы на группы:
\begin{itemize}
\item Месяцы по 31 дню: 7 штук (январь, март, май, июль, август, октябрь, декабрь).
\item Месяцы по 30 дней: 4 штуки (апрель, июнь, сентябрь, ноябрь).
\item Февраль (28 дней, в невисокосном году): 1 штука.
\end{itemize}

Обозначим:
\[
a_{31} = \frac{C_{365 - 31}^{14}}{C_{365}^{14}},
\quad
a_{30} = \frac{C_{365 - 30}^{14}}{C_{365}^{14}},
\quad
a_{28} = \frac{C_{365 - 28}^{14}}{C_{365}^{14}},
\]
и
\[
p_{31} = 1 - a_{31},
\quad
p_{30} = 1 - a_{30},
\quad
p_{28} = 1 - a_{28}.
\]

Приближённые численные значения:
\[
a_{31} \approx 0.291 \;\Rightarrow\; p_{31} \approx 0.709,
\]
\[
a_{30} \approx 0.300 \;\Rightarrow\; p_{30} \approx 0.700,
\]
\[
a_{28} \approx 0.328 \;\Rightarrow\; p_{28} \approx 0.672.
\]
Тогда
\[
\mathbb{E}[X]
= 7\,p_{31} + 4\,p_{30} + 1\,p_{28}
\approx 7 \times 0.709 \;+\; 4 \times 0.700 \;+\; 1 \times 0.672
\approx 8.44.
\]

\subsection*{Дисперсия \(\mathrm{D}[X]\)}

Дисперсия вычисляется по формуле
\[
\mathrm{D}[X]
= \mathbb{E}[X^2] - (\mathbb{E}[X])^2.
\]
При этом
\[
X^2 = \left(\sum_{m=1}^{12} I_m\right)^2
= \sum_{m=1}^{12} I_m + 2\sum_{1 \le i < j \le 12} I_i I_j,
\]
откуда
\[
\mathbb{E}[X^2]
= \sum_{m=1}^{12} \mathbb{E}[I_m]
+ 2 \sum_{1 \le i < j \le 12} \mathbb{E}[I_i I_j].
\]
Здесь \(\mathbb{E}[I_m] = p_m\) уже известны, а
\[
\mathbb{E}[I_i I_j]
= P\bigl(M_i \cap M_j\bigr),
\]
где \(M_m = \{\text{в месяце \(m\) есть хотя бы один ДР}\}\).

Чтобы найти \(P(M_i \cap M_j)\), используем
\[
P(M_i \cap M_j)
= 1 \;-\; P(\overline{M_i} \cup \overline{M_j})
= 1 \;-\; \Bigl[P(\overline{M_i}) + P(\overline{M_j}) - P(\overline{M_i} \cap \overline{M_j})\Bigr].
\]
\[
P(\overline{M_i}) = \frac{C_{365 - n_i}^{14}}{C_{365}^{14}} = a_i,
\quad
P(\overline{M_i} \cap \overline{M_j}) = \frac{C_{365 - n_i - n_j}^{14}}{C_{365}^{14}}.
\]
Тогда
\[
P(M_i \cap M_j)
= 1 - a_i - a_j + \frac{C_{365 - n_i - n_j}^{14}}{C_{365}^{14}}.
\]

Приближённо получаем для разных сочетаний (31--31, 31--30, \dots):
\[
a_{31,31} \approx \left(\frac{303}{365}\right)^{14} \approx 0.074
\quad\Longrightarrow\quad
p_{31,31} \approx 1 - 0.291 - 0.291 + 0.074 \approx 0.492,
\]
\[
a_{30,30} \approx \left(\frac{305}{365}\right)^{14} \approx 0.080
\quad\Longrightarrow\quad
p_{30,30} \approx 0.48,
\]
\[
a_{31,30} \approx \left(\frac{304}{365}\right)^{14} \approx 0.078
\quad\Longrightarrow\quad
p_{31,30} \approx 0.487,
\]
\[
a_{31,28} \approx \left(\frac{306}{365}\right)^{14} \approx 0.085
\quad\Longrightarrow\quad
p_{31,28} \approx 0.466,
\]
\[
a_{30,28} \approx \left(\frac{307}{365}\right)^{14} \approx 0.088
\quad\Longrightarrow\quad
p_{30,28} \approx 0.46.
\]

Число пар:
\[
C_{7}^{2} = 21 \;\text{(месяцы 31--31)},
\quad
C_{4}^{2} = 6 \;\text{(30--30)},
\quad
7 \times 4 = 28 \;\text{(31--30)},
\quad
7 \times 1 = 7 \;\text{(31--28)},
\quad
4 \times 1 = 4 \;\text{(30--28)}.
\]
Сумма:
\[
\sum_{i<j} p_{i,j}
\approx 21 \times 0.492 + 6 \times 0.48 + 28 \times 0.487
+ 7 \times 0.466 + 4 \times 0.46
\approx 31.95.
\]
Тогда
\[
\mathbb{E}[X^2]
= \sum_{m=1}^{12} p_m + 2 \cdot 31.95
= 8.435 + 63.9
\approx 72.335.
\]
\[
\mathrm{D}[X]
= \mathbb{E}[X^2] - (\mathbb{E}[X])^2
\approx 72.335 - (8.435)^2
\approx 72.335 - 71.11
= 1.225.
\]
Итого:
\[
\boxed{\mathbb{E}[X] \approx 8.44, \quad \mathrm{D}[X] \approx 1.23.}
\]

\section*{Решение задачи (точными выражениями)}

Пусть в группе имеется 14 человек, каждый из которых имеет \textbf{уникальный} день рождения (никакие двое не родились в один и тот же день). Считаем, что все \(365\) дней года равновероятны и независимы (исключая то, что дни рождения не совпадают). Определим случайную величину:
\[
X = \text{число месяцев, в которых есть хотя бы один день рождения}.
\]
Требуется найти:
\[
\mathbb{E}[X]
\quad \text{и} \quad
\mathrm{D}[X].
\]

\subsection*{Обозначения}

Для каждого месяца \(m \in \{1,2,\dots,12\}\) пусть \(n_m\)~--- количество дней в месяце. Тогда:
\[
n_1 = 31, \quad n_2 = 28, \quad n_3 = 31, \quad n_4 = 30, \quad
n_5 = 31, \quad n_6 = 30, \quad n_7 = 31, \quad n_8 = 31,
\]
\[
n_9 = 30, \quad n_{10} = 31, \quad n_{11} = 30, \quad n_{12} = 31.
\]
(Исходим из невисокосного года, чтобы сумма была \(365\).)

Определим событие \(M_m\): «в месяце \(m\) есть хотя бы один день рождения».
Тогда индикатор \(I_m\) этого события равен:
\[
I_m =
\begin{cases}
1, & \text{если в месяце \(m\) хотя бы один ДР},\\
0, & \text{иначе}.
\end{cases}
\]
Отсюда:
\[
X \;=\; \sum_{m=1}^{12} I_m.
\]

\subsection*{Математическое ожидание \(\mathbb{E}[X]\)}

\[
\mathbb{E}[X]
= \mathbb{E}\Bigl[\sum_{m=1}^{12} I_m\Bigr]
= \sum_{m=1}^{12} \mathbb{E}[I_m].
\]
Заметим, что
\[
\mathbb{E}[I_m] \;=\; P(M_m).
\]
А именно, \(P(M_m)\)~--- вероятность того, что хотя бы у одного из 14 человек день рождения попадает в месяц \(m\). Обозначим:
\[
P(\overline{M_m}) = \text{вероятность того, что в месяце \(m\) нет ни одного ДР}.
\]
Так как дни рождения \textbf{различны} и равновероятны по всем \(365\) дням,
\[
P(\overline{M_m}) \;=\; \frac{C_{365 - n_m}^{14}}{C_{365}^{14}},
\]
то есть нам нужно \emph{выбрать} все 14 дат рождения \emph{только} из оставшихся \(365 - n_m\) дней. Отсюда:
\[
P(M_m)
\;=\; 1 - P(\overline{M_m})
\;=\; 1 - \frac{C_{365 - n_m}^{14}}{C_{365}^{14}}.
\]
Тогда
\[
\mathbb{E}[X]
= \sum_{m=1}^{12}
\Bigl[1 - \frac{C_{365 - n_m}^{14}}{C_{365}^{14}}\Bigr].
\]
Это \textbf{точное} выражение, без каких-либо приближений.

\paragraph{Развёрнуто} (по месяцам):
\[
\mathbb{E}[X]
= \Bigl[1 - \frac{C_{334}^{14}}{C_{365}^{14}}\Bigr]
+ \Bigl[1 - \frac{C_{337}^{14}}{C_{365}^{14}}\Bigr]
+ \Bigl[1 - \frac{C_{334}^{14}}{C_{365}^{14}}\Bigr]
+ \Bigl[1 - \frac{C_{335}^{14}}{C_{365}^{14}}\Bigr]
+ \dots
\]
причём каждый месяц «вклад» в сумму даёт ровно \(1 - \frac{C_{365 - n_m}^{14}}{C_{365}^{14}}\).

\subsection*{Дисперсия \(\mathrm{D}[X]\)}

По определению дисперсии:
\[
\mathrm{D}[X]
= \mathbb{E}[X^2] - (\mathbb{E}[X])^2.
\]
Для вычисления \(\mathbb{E}[X^2]\) удобно воспользоваться формулой:
\[
X^2
= \Bigl(\sum_{m=1}^{12} I_m\Bigr)^2
= \sum_{m=1}^{12} I_m
\;+\; 2 \sum_{1 \le i < j \le 12} I_i\,I_j.
\]
Тогда
\[
\mathbb{E}[X^2]
= \sum_{m=1}^{12} \mathbb{E}[I_m]
\;+\;
2 \sum_{1 \le i < j \le 12} \mathbb{E}[I_i \, I_j].
\]
Заметим, что \(\mathbb{E}[I_m] = P(M_m)\), а
\[
\mathbb{E}[I_i \, I_j]
= P(M_i \cap M_j),
\]
то есть вероятность того, что \textbf{оба} месяца \(i\) и \(j\) содержат хотя бы один день рождения.

\paragraph{Вычисление \(P(M_i \cap M_j)\).}

\[
P(M_i \cap M_j)
= 1 - P\bigl(\overline{M_i} \cup \overline{M_j}\bigr)
= 1 - \Bigl[P(\overline{M_i}) + P(\overline{M_j}) - P(\overline{M_i} \cap \overline{M_j})\Bigr].
\]
\[
P(\overline{M_i}) = \frac{C_{365 - n_i}^{14}}{C_{365}^{14}},
\quad
P(\overline{M_j}) = \frac{C_{365 - n_j}^{14}}{C_{365}^{14}},
\]
а событие \(\overline{M_i} \cap \overline{M_j}\) означает «в месяцах \(i\) \emph{и} \(j\) нет ни одного ДР», значит
\[
P(\overline{M_i} \cap \overline{M_j})
= \frac{C_{365 - n_i - n_j}^{14}}{C_{365}^{14}}
\]
(опять же, мы выбираем все 14 дат из \(365 - n_i - n_j\) «доступных» дней, исключая месяцы \(i\) и \(j\)). Следовательно,
\[
P(M_i \cap M_j)
= 1 - \left[
\frac{C_{365 - n_i}^{14}}{C_{365}^{14}}
\;+\;
\frac{C_{365 - n_j}^{14}}{C_{365}^{14}}
\;-\;
\frac{C_{365 - n_i - n_j}^{14}}{C_{365}^{14}}
\right].
\]
Итого \(\mathbb{E}[I_i I_j] = P(M_i \cap M_j)\) будет
\[
P(M_i \cap M_j)
= 1
\;-\; \frac{C_{365 - n_i}^{14}}{C_{365}^{14}}
\;-\; \frac{C_{365 - n_j}^{14}}{C_{365}^{14}}
\;+\; \frac{C_{365 - n_i - n_j}^{14}}{C_{365}^{14}}.
\]

\paragraph{Итоговое выражение для \(\mathbb{E}[X^2]\).}

\[
\mathbb{E}[X^2]
= \sum_{m=1}^{12}
\Bigl[
1 - \frac{C_{365 - n_m}^{14}}{C_{365}^{14}}
\Bigr]
\;+\;
2 \sum_{1 \le i < j \le 12}
\Bigl[
1
\;-\; \frac{C_{365 - n_i}^{14}}{C_{365}^{14}}
\;-\; \frac{C_{365 - n_j}^{14}}{C_{365}^{14}}
\;+\; \frac{C_{365 - n_i - n_j}^{14}}{C_{365}^{14}}
\Bigr].
\]

Отсюда:
\[
\mathrm{D}[X]
=
\mathbb{E}[X^2]
\;-\;
\bigl(\mathbb{E}[X]\bigr)^2.
\]
Все слагаемые \(\binom{a}{b}\) (или \(C_a^b\)) здесь считаются точно, без приближений, и упрощать можно лишь \emph{комбинаторно}, если захочется.

\subsection*{Итоговые \textbf{точные} формулы}

\[
\boxed{
\mathbb{E}[X]
= \sum_{m=1}^{12}
\Bigl[
1 - \frac{C_{365 - n_m}^{14}}{C_{365}^{14}}
\Bigr],
\quad
\mathrm{D}[X]
= \sum_{m=1}^{12}
\Bigl[
1 - \frac{C_{365 - n_m}^{14}}{C_{365}^{14}}
\Bigr]
+ 2 \sum_{1 \le i < j \le 12}
\Bigl[
1
- \frac{C_{365 - n_i}^{14}}{C_{365}^{14}}
- \frac{C_{365 - n_j}^{14}}{C_{365}^{14}}
+ \frac{C_{365 - n_i - n_j}^{14}}{C_{365}^{14}}
\Bigr]
- \Bigl(\mathbb{E}[X]\Bigr)^2
}.
\]
Другими словами, все «числа» в этих выражениях представляют собой \emph{точные} комбинаторные величины, вычисляемые по формуле
\(
C_{n}^{k} = \binom{n}{k} = \frac{n!}{k!(\,n-k\,)!}.
\)


\end{document}