\documentclass{article}
\usepackage[utf8]{inputenc}
\usepackage[russian]{babel}
\usepackage{amsmath}
\usepackage{amssymb}
\usepackage{textcomp}


\begin{document}

\section*{Решение задачи}

На сфере радиуса $R$ равномерно распределён заряд с поверхностной плотностью $\sigma$. Потенциальная энергия взаимодействия двух точечных зарядов $q_1$ и $q_2$, находящихся в точках с радиус-векторами $\mathbf{r}_1$ и $\mathbf{r}_2$, задаётся выражением

\[
U(\mathbf{r}_1,\mathbf{r}_2) \;=\; \frac{q_1\,q_2}{|\mathbf{r}_1 - \mathbf{r}_2|^{\,1+a}}.
\]

Требуется найти силу, действующую на пробный точечный заряд $q$, расположенный внутри сферы (на расстоянии $r<R$ от её центра).

\subsection*{1. Потенциал внутри сферы}

Пусть заряд $q$ находится в точке $\mathbf{r}$ с $|\mathbf{r}|=r<R$. Поверхностный заряд равномерно распределён на сфере радиуса $R$ с плотностью $\sigma$. Элементарный вклад в потенциал в точке $\mathbf{r}$ от элемента площади $dS'$ в точке $\mathbf{r}'$ сферы (где $|\mathbf{r}'|=R$) равен
\[
d\Phi(\mathbf{r}) \;=\; \frac{\sigma\,dS'}{|\mathbf{r} - \mathbf{r}'|^{\,1+a}}.
\]
Тогда полный потенциал:
\[
\Phi(\mathbf{r}) \;=\; \sigma \int_{\text{сфера}} \frac{dS'}{|\mathbf{r} - \mathbf{r}'|^{\,1+a}}.
\]

Благодаря сферической симметрии, удобно выбрать ось $z$ вдоль вектора $\mathbf{r}$. Тогда $|\mathbf{r}|=r$, для точек на сфере $|\mathbf{r}'|=R$, а угол $\theta'$ — угол между $\mathbf{r}$ и $\mathbf{r}'$. Запишем:

\[
|\mathbf{r}-\mathbf{r}'|
= \sqrt{\,R^2 + r^2 - 2\,R\,r\,\cos\theta'}.
\]
Площадь элемента сферы: $dS' = R^2\sin\theta'\, d\theta'\, d\phi'$. Тогда
\[
\Phi(r)
= \sigma \int_0^{2\pi}\! \int_0^\pi
\frac{R^2\,\sin\theta'\,d\theta'\,d\phi'}{\bigl(R^2 + r^2 - 2Rr\cos\theta'\bigr)^{\frac{1+a}{2}}}.
\]
Интеграция по $\phi'$ даёт множитель $2\pi$. Получаем
\[
\Phi(r)
= 2\pi\,\sigma\,R^2
\int_0^\pi
\frac{\sin\theta'\, d\theta'}{\bigl(R^2 + r^2 - 2Rr\cos\theta'\bigr)^{\frac{1+a}{2}}}.
\]

Введём $\alpha = \frac{1+a}{2}$. Сменим переменную $t = \cos\theta'$, тогда $\sin\theta'\, d\theta' = -\,dt$, и границы по $t$ изменяются от $1$ до $-1$. В результате:

\[
\Phi(r)
= 2\pi\,\sigma\,R^2
\int_{-1}^1
\frac{dt}{\bigl(R^2 + r^2 - 2Rr\,t\bigr)^{\alpha}}.
\]
Обозначим $x = R^2 + r^2 - 2Rr\,t$, тогда $dx = -2Rr\,dt$. При $t=-1$ будет $x=(R+r)^2$, при $t=1$ — $x=(R-r)^2$. Значит

\[
\Phi(r)
= 2\pi\,\sigma\,R^2
\int_{(R+r)^2}^{(R-r)^2}
\frac{-\,dx}{2Rr\,x^{\alpha}}
= \frac{\pi\,\sigma\,R^2}{R\,r}
\int_{(R+r)^2}^{(R-r)^2}
x^{-\alpha}\,dx.
\]
Учитывая $\alpha = \tfrac{1+a}{2}$ и интеграл
\[
\int x^{-\alpha}\,dx = \int x^{-\frac{1+a}{2}}\,dx
=
\begin{cases}
\dfrac{2}{\,1-a\,}\,x^{\frac{1-a}{2}}, & a \neq 1,\\
\ln x, & a=1,
\end{cases}
\]
для $a \neq 1$ получаем:

\[
\Phi(r)
= \frac{\pi\,\sigma\,R^2}{R\,r}
\cdot \frac{2}{1-a}
\Bigl[
\,\bigl((R-r)^2\bigr)^{\frac{1-a}{2}}
- \bigl((R+r)^2\bigr)^{\frac{1-a}{2}}
\Bigr].
\]
С учётом порядка пределов эта запись эквивалентна более удобному виду:

\[
\Phi(r)
= \frac{2\pi\,\sigma\,R}{(1-a)\,r}\,\Bigl[
(R+r)^{\,1-a} - (R-r)^{\,1-a}
\Bigr],\quad r<R,\;a\neq 1.
\]
При $a=0$ (обычный кулоновский случай $1/r$) внутри равномерно заряженной сферы потенциал оказывается константой, а поле (и сила) — равным нулю.

\subsection*{2. Сила на пробный заряд $q$}

Сила на заряд $q$ равна
\[
\mathbf{F}(r) = -\,q\,\nabla \Phi(r).
\]
Из-за симметрии $\mathbf{F}(r)$ направлена вдоль радиуса $\mathbf{r}$. Пусть $\hat{\mathbf{r}} = \frac{\mathbf{r}}{r}$. Тогда

\[
\mathbf{F}(r)
= F(r)\,\hat{\mathbf{r}},\quad
F(r)
= -\,q\,\frac{d\Phi(r)}{dr}.
\]
С учётом
\[
\Phi(r)
= \frac{2\pi\,\sigma\,R}{1-a}\,\frac{(R+r)^{1-a} - (R-r)^{1-a}}{r},
\]
получаем

\[
F(r)
= -\,q\,\frac{d}{dr}
\Biggl[
\frac{2\pi\,\sigma\,R}{1-a}\,\frac{(R+r)^{1-a} - (R-r)^{1-a}}{r}
\Biggr].
\]
Раскрывая производную, можно оставить ответ в виде:
\[
\mathbf{F}(r)
= -\,q\,\nabla\Phi(r)
= -\,q\,\frac{2\pi\,\sigma\,R}{1-a}
\,\frac{d}{dr}
\Bigl[
\frac{(R+r)^{\,1-a} - (R-r)^{\,1-a}}{\,r\,}
\Bigr]
\,\hat{\mathbf{r}},\quad r < R,\; a \neq 1.
\]
Либо выписать полностью:

\[
\mathbf{F}(r)
= -\,q\,2\pi\,\sigma\,R
\Biggl[
\frac{(R+r)^{-a} + (R-r)^{-a}}{r}
\;-\;
\frac{1}{\,1-a\,}\,\frac{(R+r)^{1-a} - (R-r)^{1-a}}{r^2}
\Biggr]
\hat{\mathbf{r}}.
\]
В частности, в центре сферы ($r=0$) сила равна нулю, что согласуется с физической симметрией задачи.

\subsection*{3. Итоговый ответ}

\[
\boxed{
\mathbf{F}(r)
=
-\,q\,\nabla \Phi(r)
=
-\,q\;\frac{2\pi\,\sigma\,R}{\,1-a\,}\,
\frac{d}{dr}
\Bigl[
\frac{(R+r)^{1-a} \;-\; (R-r)^{1-a}}{\,r\,}
\Bigr]\;\hat{\mathbf{r}},
\quad r<R,\;a\neq1.
}
\]

\noindent
При $a=0$ (обычном кулоновском законе) сила внутри сферы равна нулю. При $a=1$ требуется отдельное рассмотрение интеграла с логарифмом, однако метод решения аналогичен.


\end{document}