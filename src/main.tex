\documentclass{article}
\usepackage[utf8]{inputenc}
\usepackage[russian]{babel}
\usepackage{amsmath}
\usepackage{amssymb}
\usepackage{textcomp}

\begin{document}

Задача \textit{Необычная планета}:

При какой форме однородной планеты объёма $V$, состоящей из несжимаемого вещества, сила тяжести на её поверхности будет максимальна (в какой-то точке) среди всех возможных форм?
(Подсказка: можно считать, что планета односвязна и имеет осевую симметрию.)

Решение:
Рассмотрим задачу в цилиндрических координатах. Ось $z$ проходить через точку наблюдения и центр тяжести. Точку наблюдения будем считать началом системы отсчета.

Точка в которой мы рассматриваем поле будет находится на оси симметрии. Ведь если это не так, то есть облась планеты с коорденатой $z_0$, которая не симметрична по углу $\varphi$.

Разместив материя с этой координатой $z_0$ симметрично по $\varphi$ мы уменьшим расстояние до точки наблюдения и увеличим $\cos \theta$, что увеличит поле при том же объеме планеты.

Получаем: ось $z$ - ось симметрии планеты.

Будем искать $R(z)$ - форму планеты.
\par
\par 1) У нас есть ограничение на объем (так как мы не знаем ограничений на $z$, то пока формально интеграл будет до $+\infty$):
\[
 \int\limits_0^{+\infty} \int\limits_0^{R(z)} \int\limits_0^{2 \pi} R dz dR d\varphi=V \Rightarrow \int\limits_0^{+\infty} \frac{R^2(z)}{2}dz= \frac{V}{2 \pi}
\]
\par
\par Разделим планету на тонкие диски и будем суммировать поле от дисков в интересующей нас точке.
\[
E_{disk} = 2\pi \sigma \left( 1 - \frac{z}{\sqrt{z^2 + R^2(z)}} \right) = 2\pi \rho dz \left( 1 - \frac{z}{\sqrt{z^2 + R^2(z)}} \right)
\]
\[
E = 2\pi \rho \int\limits_0^{+\infty} \left( 1 - \frac{z}{\sqrt{z^2 + R^2(z)}} \right) dz \textrightarrow sup
\]

Будем искать форму планеты вариационным методом, с учетом условия трансверсальности (конечности объема):
\[
 F = \left( 1 - \frac{z}{\sqrt{z^2 + R^2(z)}} \right) -\lambda \frac{R^2(z)}{2}
\]
Уравнение Эйлера-Лагранжа:
\[
 \frac{\partial F}{\partial R} - \frac{d}{dz} \frac{\partial F}{\partial R'} = 0 \Rightarrow \frac{\partial F}{\partial R} = 0
\]
\[
 \frac{z \cdot R}{\left( z^2 + R^2 \right)^{3/2}} - \lambda R = 0  \Rightarrow R(z) = \sqrt{\left( \frac{z}{\lambda} \right)^{2/3} - z^2}
\]
Так как $R(z) > 0$, получаем что $z_{max} = \sqrt{\frac{1}{\lambda}}$.
\par Отсюда получаем уравнение на $\lambda$:
\[
 \int\limits_0^{\sqrt{\frac{1}{\lambda}}} \left( \left( \frac{z}{\lambda} \right)^{2/3} - z^2 \right)dz= \frac{V}{\pi} \Rightarrow \lambda = \left( \frac{15}{4 \pi} V \right)^{-2/3}
\]
Таким образом ответ:
\[
 R(z) = \sqrt{\left( \frac{z}{\lambda} \right)^{2/3} - z^2} \, , \,\,  \lambda = \left( \frac{15}{4 \pi} V \right)^{-2/3}
\]
\end{document}