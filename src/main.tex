\documentclass{article}
\usepackage[utf8]{inputenc}
\usepackage[russian]{babel}
\usepackage{amsmath}
\usepackage{amssymb}
\usepackage{textcomp}


\begin{document}

\section*{Решение задачи о ёмкости шара с диэлектрическим слоем}

\textbf{Папочка}, ниже я подробно расписываю вывод формулы для электроёмкости металлического шара радиуса \(R\), который покрыт слоем диэлектрика толщиной \(d\) с диэлектрической проницаемостью \(\varepsilon\). За пределами диэлектрика (то есть при \(r > R + d\)) у нас вакуум (\(\varepsilon_0\)).

\subsection*{1. Постановка задачи}
Пусть металлический шар радиуса \(R\) заряжен зарядом \(Q\). Потенциал шара обозначим \(V\). На бесконечности потенциал полагаем равным нулю, то есть \(V(\infty) = 0\). Нужно найти электроёмкость:
\[
C = \frac{Q}{V}.
\]
Без покрытия диэлектриком известна классическая формула (для шара в вакууме):
\[
C_{\text{вак}} = 4 \pi \varepsilon_0 R.
\]
Теперь же прибавляется слой диэлектрика с проницаемостью \(\varepsilon\) толщиной \(d\).

\subsection*{2. Схема решения}
В сферически-симметричной задаче электрическое поле вне шара, в слое диэлектрика и в вакууме за пределами слоя имеет вид:
\[
E(r) =
\begin{cases}
0, & 0 \leq r < R \quad (\text{металл, }E=0\text{ внутри проводника}), \\
\dfrac{Q}{4\pi\varepsilon_0 \varepsilon \, r^2}, & R \leq r \leq R+d \quad (\text{диэлектрик}), \\
\dfrac{Q}{4\pi\varepsilon_0 \, r^2}, & r > R + d \quad (\text{вакуум}).
\end{cases}
\]
Полный потенциал шара (при условии \(V(\infty)=0\)) будет равен сумме потенциалов, набираемых от \(r=R\) до \(r=R+d\) (участок диэлектрика) и далее от \(r=R+d\) до бесконечности (участок вакуума).

\[
V = \int_{r=R}^{r=R+d} E_{\text{диэл}}(r)\,dr
   + \int_{r=R+d}^{\infty} E_{\text{внеш}}(r)\,dr.
\]

\subsection*{3. Расчёт вклада диэлектрика}
На участке \(R \le r \le R+d\):
\[
E_{\text{диэл}}(r)
= \frac{Q}{4\pi \varepsilon_0 \varepsilon \, r^2}.
\]
Тогда вклад в потенциал:
\[
\Delta V_{\text{диэл}}
= \int_{R}^{R+d} \frac{Q}{4\pi \varepsilon_0 \varepsilon\,r^2}\,dr
= \frac{Q}{4\pi \varepsilon_0 \varepsilon}
\int_{R}^{R+d} \frac{dr}{r^2}.
\]
Интеграл \(\int \frac{dr}{r^2} = -\frac{1}{r}\). Поэтому:
\[
\Delta V_{\text{диэл}}
= \frac{Q}{4\pi \varepsilon_0 \varepsilon}
\left[-\frac{1}{r}\right]_{r=R}^{r=R+d}
= \frac{Q}{4\pi \varepsilon_0 \varepsilon}
\left(\frac{1}{R} - \frac{1}{R + d}\right).
\]

\subsection*{4. Расчёт вклада внешней области (вакуум)}
Для \(r > R + d\) имеем
\[
E_{\text{внеш}}(r) = \frac{Q}{4\pi \varepsilon_0\,r^2}.
\]
Тогда:
\[
\Delta V_{\text{внеш}}
= \int_{R+d}^{\infty} \frac{Q}{4\pi \varepsilon_0\,r^2}\,dr
= \frac{Q}{4\pi \varepsilon_0}\int_{R+d}^{\infty} \frac{dr}{r^2}
= \frac{Q}{4\pi \varepsilon_0}
\left[-\frac{1}{r}\right]_{r=R+d}^{\infty}
= \frac{Q}{4\pi \varepsilon_0} \cdot \frac{1}{R + d}.
\]

\subsection*{5. Суммирование потенциалов}
Таким образом, полный потенциал шара:
\[
V
= \Delta V_{\text{диэл}} + \Delta V_{\text{внеш}}
= \frac{Q}{4\pi \varepsilon_0 \varepsilon}\,\biggl(\frac{1}{R} - \frac{1}{R + d}\biggr)
+ \frac{Q}{4\pi \varepsilon_0}\,\frac{1}{R + d}.
\]
Вынесем общий множитель \(\displaystyle \frac{Q}{4\pi \varepsilon_0}\):
\[
V
= \frac{Q}{4\pi \varepsilon_0}
\left[
\frac{1}{\varepsilon}
\left(\frac{1}{R} - \frac{1}{R + d}\right)
+ \frac{1}{R + d}
\right].
\]
Проведём алгебраические упрощения и в итоге получим:
\[
V
= \frac{Q}{4\pi \varepsilon_0}
\cdot
\frac{\varepsilon\,R + \varepsilon\,d - d}{\varepsilon\,R\,\bigl(R + d\bigr)}.
\]
А значит
\[
V
= \frac{Q}{4\pi \varepsilon_0}
\cdot
\frac{\varepsilon\,(R + d) - d}{\varepsilon\,R\,(R + d)}
= \frac{Q}{4\pi \varepsilon_0}
\cdot
\frac{\varepsilon(R + d) - d}{\varepsilon R (R + d)}.
\]
Тогда искомая ёмкость \(C = \dfrac{Q}{V}\) будет
\[
\boxed{
C
= 4\pi \varepsilon_0\,\frac{\varepsilon\,R\,(R + d)}{\varepsilon\,R + d}.
}
\]
Это и есть окончательная формула для электроёмкости металлического шара радиуса \(R\), покрытого диэлектриком толщиной \(d\) и диэлектрической проницаемостью \(\varepsilon\).

\end{document}