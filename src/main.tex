\documentclass{article}
\usepackage{amsmath}
\usepackage{amsfonts}
\usepackage{amssymb}
\usepackage[russian]{babel}
\usepackage[utf8]{inputenc}

\begin{document}

\title{Решение задачи 1.1}
\date{}
\maketitle

\section*{1.1.}

Необходимо предложить две эквивалентные схемы реализации данного двухполюсника двумя элементами, соединенными:

\begin{itemize}
    \item а) последовательно;
    \item б) параллельно.
\end{itemize}

\subsection*{Дано:}

Напряжение:
\[
u(t) = 80 + 2\cos(200t - 15^\circ) \, \text{В}
\]

Ток:
\[
i(t) = \sin(200t - 60^\circ) \, \text{А}
\]

\subsection*{Решение:}

\subsubsection*{Шаг 1: Преобразование тока к косинусной форме}

Преобразуем ток для удобства сравнения:

\[
i(t) = \sin(200t - 60^\circ) = \cos\left(200t - 60^\circ - 90^\circ\right) = \cos(200t - 150^\circ) \, \text{А}
\]

\subsubsection*{Шаг 2: Выделение постоянной и переменной составляющих}

Напряжение содержит постоянную составляющую \( U_{\text{DC}} = 80 \, \text{В} \) и переменную составляющую \( U_{\text{AC}} = 2\cos(200t - 15^\circ) \, \text{В} \). Ток является чисто переменным.

\subsubsection*{Шаг 3: Представление в комплексной форме (фазоры)}

Для анализа переменной составляющей используем фазоры:

\[
\underline{U} = 2\angle (-15^\circ) \, \text{В}
\]
\[
\underline{I} = 1\angle (-150^\circ) \, \text{А}
\]

\subsubsection*{Шаг 4: Вычисление импеданса}

Используя закон Ома в комплексной форме:

\[
\underline{Z} = \frac{\underline{U}}{\underline{I}} = \frac{2\angle (-15^\circ)}{1\angle (-150^\circ)} = 2\angle (135^\circ)
\]

Преобразуем импеданс в алгебраическую форму:

\[
\underline{Z} = 2\left( \cos 135^\circ + j \sin 135^\circ \right) = 2\left( -\frac{\sqrt{2}}{2} + j\frac{\sqrt{2}}{2} \right) = -1.4142 + j\,1.4142 \, \Omega
\]

\textbf{Примечание:} Отрицательная действительная часть импеданса указывает на отрицательное сопротивление, что предполагает наличие активных элементов.

\subsubsection*{Вариант а) Последовательное соединение}

\paragraph*{Элементы схемы:}

\begin{itemize}
    \item \textbf{Источник напряжения} \( V_s = 80 \, \text{В} \) (постоянная составляющая)
    \item \textbf{Сопротивление} \( R_s = -1.4142 \, \Omega \)
    \item \textbf{Индуктивность} \( L_s \)
\end{itemize}

\paragraph*{Вычисление индуктивности:}

Зная, что \( X_s = \omega L_s \), где \( \omega = 200 \, \text{рад/с} \):

\[
X_s = 1.4142 \, \Omega
\]
\[
L_s = \frac{X_s}{\omega} = \frac{1.4142}{200} = 0.00707 \, \text{Гн}
\]

\paragraph*{Итоговая последовательная схема:}

\begin{itemize}
    \item \textbf{Источник напряжения:} \( V_s = 80 \, \text{В} \)
    \item \textbf{Отрицательное сопротивление:} \( R_s = -1.4142 \, \Omega \)
    \item \textbf{Индуктивность:} \( L_s = 7.07 \, \text{мГн} \)
\end{itemize}

\subsubsection*{Вариант б) Параллельное соединение}

\paragraph*{Вычисление проводимости:}

Сначала находим комплексную проводимость \( \underline{Y} \):

\[
\underline{Y} = \frac{1}{\underline{Z}} = \frac{1}{2\angle 135^\circ} = 0.5\angle (-135^\circ)
\]

Преобразуем проводимость в алгебраическую форму:

\[
\underline{Y} = 0.5\left( \cos(-135^\circ) + j \sin(-135^\circ) \right) = 0.5\left( -\frac{\sqrt{2}}{2} - j\frac{\sqrt{2}}{2} \right) = -0.3536 - j\,0.3536 \, \text{См}
\]

\paragraph*{Элементы схемы:}

\begin{itemize}
    \item \textbf{Проводимость} \( G_p = -0.3536 \, \text{См} \)
    \item \textbf{Сусцептанс} \( B_p = -0.3536 \, \text{См} \)
\end{itemize}

\paragraph*{Вычисление емкости:}

Так как \( B_p = -\omega C_p \), то:

\[
C_p = \frac{-B_p}{\omega} = \frac{0.3536}{200} = 0.001768 \, \text{Ф}
\]

\paragraph*{Итоговая параллельная схема:}

\begin{itemize}
    \item \textbf{Отрицательное сопротивление:} \( R_p = \frac{1}{G_p} = -2.8284 \, \Omega \)
    \item \textbf{Емкость:} \( C_p = 1.768 \, \text{мФ}
\end{itemize}

\subsection*{Вывод:}

\begin{itemize}
    \item \textbf{Вариант а)}: Двухполюсник можно представить как последовательное соединение источника напряжения \( 80 \, \text{В} \), отрицательного сопротивления \( -1.4142 \, \Omega \) и индуктивности \( 7.07 \, \text{мГн} \).
    \item \textbf{Вариант б)}: Альтернативно, его можно представить как параллельное соединение отрицательного сопротивления \( -2.8284 \, \Omega \) и емкости \( 1.768 \, \text{мФ} \).
\end{itemize}

\textbf{Важно:} Отрицательное сопротивление и проводимость указывают на наличие активных элементов (например, управляемых источников или усилителей). В реальных цепях пассивные компоненты не могут иметь отрицательное сопротивление. Поэтому реализация таких схем требует использования активных элементов.

\end{document}