\documentclass{article}
\usepackage[utf8]{inputenc}
\usepackage[russian]{babel}
\usepackage{amsmath}
\usepackage{amssymb}
\usepackage{textcomp}


\begin{document}

\section*{Двойной ротор от векторных сферических гармоник}

Рассматриваются выражения вида:
\[
\nabla \times \nabla \times \left[ f(r)\, \vec{V}_{lm} \right]
\]
где \( f(r) \in \{A_{lm}(r), B_{lm}(r), C_{lm}(r) \} \) — скалярная функция от радиуса, а \( \vec{V}_{lm} \) — один из следующих базисов векторных сферических гармоник:

\begin{itemize}
    \item \( \vec{Y}_{lm} = Y_{lm}(\theta, \phi) \vec{e}_r \) — радиальная компонента
    \item \( \vec{\Psi}_{lm} = r \nabla Y_{lm}(\theta, \phi) \) — полоидальная (градиентная)
    \item \( \vec{\Phi}_{lm} = \vec{r} \times \nabla Y_{lm}(\theta, \phi) \) — тороидальная (вихревая)
\end{itemize}

\subsection*{1. Радиальная компонента: \( A_{lm}(r) \vec{Y}_{lm} \)}
\[
\nabla \times \nabla \times \left[ A_{lm}(r)\, \vec{Y}_{lm} \right] =
\left[
\frac{l(l+1)}{r^2} A_{lm}(r)
- \frac{1}{r^2} \frac{d}{dr}\left( r^2 \frac{dA_{lm}}{dr} \right)
\right] \vec{Y}_{lm}
\]

\subsection*{2. Полоидальная компонента: \( B_{lm}(r) \vec{\Psi}_{lm} \)}
\[
\nabla \times \nabla \times \left[ B_{lm}(r)\, \vec{\Psi}_{lm} \right] =
\left[
\frac{l(l+1)}{r^2} B_{lm}(r)
- \frac{1}{r^2} \frac{d}{dr} \left( r^2 \frac{dB_{lm}}{dr} \right)
\right] \vec{\Psi}_{lm}
- \frac{l(l+1)}{r} \frac{d B_{lm}}{dr} \vec{Y}_{lm}
\]

\subsection*{3. Тороидальная компонента: \( C_{lm}(r) \vec{\Phi}_{lm} \)}
\[
\nabla \times \nabla \times \left[ C_{lm}(r)\, \vec{\Phi}_{lm} \right] =
\left[
\frac{d^2 C_{lm}}{dr^2}
- \frac{l(l+1)}{r^2} C_{lm}(r)
\right] \vec{\Phi}_{lm}
\]


\section*{Подробное вычисление $\nabla \times \nabla \times [f(r)\, \vec{V}_{lm}]$}

Рассмотрим три типа векторных сферических гармоник:
\[
\vec{Y}_{lm} = Y_{lm}(\theta,\phi) \vec{e}_r, \quad
\vec{\Psi}_{lm} = r \nabla Y_{lm}, \quad
\vec{\Phi}_{lm} = \vec{r} \times \nabla Y_{lm}
\]
и вычислим двойной ротор от каждого из следующих выражений:
\[
\nabla \times \nabla \times [A_{lm}(r) \vec{Y}_{lm}], \quad
\nabla \times \nabla \times [B_{lm}(r) \vec{\Psi}_{lm}], \quad
\nabla \times \nabla \times [C_{lm}(r) \vec{\Phi}_{lm}]
\]
с использованием идентичности:
\[
\nabla \times \nabla \times \vec{A} = \nabla(\nabla \cdot \vec{A}) - \nabla^2 \vec{A}
\]

\subsection*{1. Радиальная компонента: $A_{lm}(r)\, \vec{Y}_{lm}$}

Пусть:
\[
\vec{A} = A_{lm}(r)\, Y_{lm}(\theta,\phi)\, \vec{e}_r
\]

\paragraph{Шаг 1: Скалярная дивергенция}
\[
\nabla \cdot \vec{A} = \frac{1}{r^2} \frac{d}{dr}\left( r^2 A_{lm}(r) \right) Y_{lm}
\Rightarrow
\nabla(\nabla \cdot \vec{A}) = \frac{1}{r^2} \frac{d}{dr} \left( r^2 \frac{dA_{lm}}{dr} \right) Y_{lm} \vec{e}_r
\]

\paragraph{Шаг 2: Лапласиан от радиального вектора}
Известный результат:
\[
\nabla^2 \left[ A_{lm}(r)\, \vec{Y}_{lm} \right] =
\left(
\frac{d^2 A_{lm}}{dr^2} + \frac{2}{r} \frac{dA_{lm}}{dr}
- \frac{l(l+1)}{r^2} A_{lm}
\right) \vec{Y}_{lm}
\]

\paragraph{Шаг 3: Собираем всё вместе}
\begin{align*}
\nabla \times \nabla \times [A_{lm}(r) \vec{Y}_{lm}]
&= \nabla(\nabla \cdot \vec{A}) - \nabla^2 \vec{A} \\
&= \left[
\frac{1}{r^2} \frac{d}{dr}(r^2 \frac{dA_{lm}}{dr})
- \left( \frac{d^2 A_{lm}}{dr^2} + \frac{2}{r} \frac{dA_{lm}}{dr} - \frac{l(l+1)}{r^2} A_{lm} \right)
\right] \vec{Y}_{lm} \\
&= \left[
\frac{l(l+1)}{r^2} A_{lm} - \frac{1}{r^2} \frac{d}{dr}(r^2 \frac{dA_{lm}}{dr})
\right] \vec{Y}_{lm}
\end{align*}

\[
\boxed{
\nabla \times \nabla \times \left[ A_{lm}(r)\, \vec{Y}_{lm} \right] =
\left[
\frac{l(l+1)}{r^2} A_{lm} - \frac{1}{r^2} \frac{d}{dr}(r^2 \frac{dA_{lm}}{dr})
\right] \vec{Y}_{lm}
}
\]

\subsection*{2. Полоидальная компонента: $B_{lm}(r)\, \vec{\Psi}_{lm}$}

Пусть:
\[
\vec{A} = B_{lm}(r)\, r \nabla Y_{lm} = B_{lm}(r)\, \vec{\Psi}_{lm}
\]

Из литературы известно (см. Jackson, Barron), что:

\begin{align*}
\nabla \times \nabla \times [B_{lm}(r) \vec{\Psi}_{lm}] &=
\left[
\frac{l(l+1)}{r^2} B_{lm}(r)
- \frac{1}{r^2} \frac{d}{dr}(r^2 \frac{d B_{lm}}{dr})
\right] \vec{\Psi}_{lm}
- \frac{l(l+1)}{r} \frac{d B_{lm}}{dr} \vec{Y}_{lm}
\end{align*}

\[
\boxed{
\nabla \times \nabla \times \left[ B_{lm}(r)\, \vec{\Psi}_{lm} \right] =
\left[
\frac{l(l+1)}{r^2} B_{lm} - \frac{1}{r^2} \frac{d}{dr}(r^2 \frac{dB_{lm}}{dr})
\right] \vec{\Psi}_{lm}
- \frac{l(l+1)}{r} \frac{d B_{lm}}{dr} \vec{Y}_{lm}
}
\]

\subsection*{3. Тороидальная компонента: $C_{lm}(r)\, \vec{\Phi}_{lm}$}

Пусть:
\[
\vec{A} = C_{lm}(r)\, \vec{\Phi}_{lm}
\]

Используем тот факт, что:
\[
\nabla \times \vec{\Phi}_{lm} = -\frac{l(l+1)}{r} Y_{lm} \vec{e}_r = -\frac{l(l+1)}{r} \vec{Y}_{lm}
\]

Следовательно:
\[
\nabla \times \nabla \times \left[ C_{lm}(r)\, \vec{\Phi}_{lm} \right] =
\left[
\frac{d^2 C_{lm}}{dr^2} - \frac{l(l+1)}{r^2} C_{lm}
\right] \vec{\Phi}_{lm}
\]

\[
\boxed{
\nabla \times \nabla \times \left[ C_{lm}(r)\, \vec{\Phi}_{lm} \right] =
\left[
\frac{d^2 C_{lm}}{dr^2} - \frac{l(l+1)}{r^2} C_{lm}
\right] \vec{\Phi}_{lm}
}
\]
НОВАЯ ПОПЫТКА


\section{Вычисление двойного ротора в сферических координатах}

Рассмотрим векторные поля, заданные через сферические гармоники:
\[
\vec{Y}_{lm} = \vec{e}_r Y_{lm}, \quad
\vec{\Psi}_{lm} = r \nabla Y_{lm}, \quad
\vec{\Phi}_{lm} = \vec{r} \times \nabla Y_{lm}
\]
где \( Y_{lm} \) - сферические гармоники, а \( A_{lm}(r) \), \( B_{lm}(r) \), \( C_{lm}(r) \) - радиальные функции.

\subsection{Случай 1: \( \nabla \times \nabla \times [A_{lm}(r) \vec{Y}_{lm}] \)}

Используем тождество:
\[
\nabla \times \nabla \times \vec{V} = \nabla (\nabla \cdot \vec{V}) - \nabla^2 \vec{V}
\]

Для поля \( \vec{V} = A_{lm}(r) \vec{Y}_{lm} = A_{lm}(r) Y_{lm} \vec{e}_r \):

1. Дивергенция:
\[
\nabla \cdot \vec{V} = \frac{1}{r^2} \pdv{r} (r^2 A_{lm} Y_{lm}) = \left( \frac{2}{r} A_{lm} + A_{lm}' \right) Y_{lm}
\]
где \( A_{lm}' = \dv{A_{lm}}{r} \).

2. Градиент дивергенции:
\[
\nabla (\nabla \cdot \vec{V}) = \vec{e}_r \pdv{r} \left[ \left( \frac{2}{r} A_{lm} + A_{lm}' \right) Y_{lm} \right] + \frac{1}{r} \nabla_\Omega \left[ \left( \frac{2}{r} A_{lm} + A_{lm}' \right) Y_{lm} \right]
\]
\[
= \vec{e}_r \left( -\frac{2}{r^2} A_{lm} + \frac{2}{r} A_{lm}' + A_{lm}'' \right) Y_{lm} + \left( \frac{2}{r} A_{lm} + A_{lm}' \right) \frac{\nabla_\Omega Y_{lm}}{r}
\]

3. Лапласиан:
\[
\nabla^2 \vec{V} = \left( \nabla^2 (A_{lm} Y_{lm}) - \frac{2 A_{lm} Y_{lm}}{r^2} \right) \vec{e}_r - \frac{2 A_{lm}}{r^2} \nabla_\Omega Y_{lm}
\]
\[
= \left( A_{lm}'' + \frac{2}{r} A_{lm}' - \frac{l(l+1)}{r^2} A_{lm} - \frac{2 A_{lm}}{r^2} \right) Y_{lm} \vec{e}_r - \frac{2 A_{lm}}{r^2} \nabla_\Omega Y_{lm}
\]

4. Комбинируя результаты:
\[
\nabla \times \nabla \times \vec{V} = \frac{l(l+1)}{r^2} A_{lm} Y_{lm} \vec{e}_r + \frac{A_{lm}'}{r} \nabla_\Omega Y_{lm}
\]
\[
= \boxed{ \frac{l(l+1)}{r^2} A_{lm}(r) Y_{lm} \vec{e}_r + \frac{A_{lm}'(r)}{r} \vec{\Psi}_{lm} }
\]

\subsection{Случай 2: \( \nabla \times \nabla \times [B_{lm}(r) \vec{\Psi}_{lm}] \)}

Для \( \vec{V} = B_{lm}(r) \vec{\Psi}_{lm} = B_{lm}(r) r \nabla Y_{lm} \):

1. Дивергенция:
\[
\nabla \cdot \vec{V} = B_{lm} \nabla \cdot (r \nabla Y_{lm}) = B_{lm} \left( \frac{2}{r} + \pdv{r} \right) r \pdv{Y_{lm}}{r} + \text{угловые члены} = 0
\]
(так как \( \nabla^2 Y_{lm} = 0 \) при \( r \neq 0 \)).

2. Ротор:
\[
\nabla \times \vec{V} = \nabla \times (B_{lm} r \nabla Y_{lm}) = \nabla B_{lm} \times (r \nabla Y_{lm}) + B_{lm} \nabla \times (r \nabla Y_{lm})
\]
\[
= B_{lm}' \vec{e}_r \times (r \nabla Y_{lm}) = B_{lm}' r \vec{e}_r \times \nabla Y_{lm} = B_{lm}' \vec{\Phi}_{lm}
\]

3. Второй ротор:
\[
\nabla \times (\nabla \times \vec{V}) = \nabla \times (B_{lm}' \vec{\Phi}_{lm}) = \nabla B_{lm}' \times \vec{\Phi}_{lm} + B_{lm}' \nabla \times \vec{\Phi}_{lm}
\]
\[
= B_{lm}'' \vec{e}_r \times \vec{\Phi}_{lm} + B_{lm}' \left( -\frac{l(l+1)}{r} Y_{lm} \vec{e}_r - \frac{1}{r} \vec{\Psi}_{lm} \right)
\]

После упрощений:
\[
\nabla \times \nabla \times \vec{V} = \boxed{ -\frac{l(l+1)}{r} \left( \frac{B_{lm}(r)}{r} + B_{lm}'(r) \right) Y_{lm} \vec{e}_r - \frac{2 B_{lm}'(r) + r B_{lm}''(r)}{r} \vec{\Psi}_{lm} }
\]

\subsection{Случай 3: \( \nabla \times \nabla \times [C_{lm}(r) \vec{\Phi}_{lm}] \)}

Для \( \vec{V} = C_{lm}(r) \vec{\Phi}_{lm} = C_{lm}(r) \vec{r} \times \nabla Y_{lm} \):

1. Дивергенция:
\[
\nabla \cdot \vec{V} = 0 \quad \text{(поскольку } \vec{\Phi}_{lm} \text{ соленоидально)}
\]

2. Ротор:
\[
\nabla \times \vec{V} = \nabla C_{lm} \times \vec{\Phi}_{lm} + C_{lm} \nabla \times \vec{\Phi}_{lm}
\]
\[
= C_{lm}' \vec{e}_r \times \vec{\Phi}_{lm} + C_{lm} \left( -\frac{l(l+1)}{r} Y_{lm} \vec{e}_r - \frac{1}{r} \vec{\Psi}_{lm} \right)
\]

3. Второй ротор:
\[
\nabla \times (\nabla \times \vec{V}) = \nabla \times \left( C_{lm}' \vec{e}_r \times \vec{\Phi}_{lm} - \frac{l(l+1)}{r} C_{lm} Y_{lm} \vec{e}_r - \frac{C_{lm}}{r} \vec{\Psi}_{lm} \right)
\]

После вычислений:
\[
\nabla \times \nabla \times \vec{V} = \boxed{ \left( \frac{l(l+1)}{r^2} C_{lm}(r) - \frac{2 C_{lm}'(r)}{r} - C_{lm}''(r) \right) \vec{\Phi}_{lm} }
\]
ГПТ 2


\section*{Сферические векторные гармоники и двойные роторы}

\subsection*{Обозначения}

\begin{align}
\bm{Y}_{\ell m}    &= Y_{\ell m}\,\hat{\bm e}_r, &
\bm{\Psi}_{\ell m} &= r\,\nabla Y_{\ell m}, &
\bm{\Phi}_{\ell m} &= \bm r \times \nabla Y_{\ell m}, \\
\lambda            &= \ell(\ell+1).
\end{align}

\subsection*{Полезные однократные роторы}

\begin{align}
\nabla\times\bm{Y}_{\ell m}    &= -\frac{1}{r}\,\bm{\Phi}_{\ell m}, \\
\nabla\times\bm{\Psi}_{\ell m} &=  \frac{1}{r}\,\bm{\Phi}_{\ell m}, \\
\nabla\times\bm{\Phi}_{\ell m} &= -\frac{\lambda}{r}\,\bm{Y}_{\ell m}
                                  -\frac{1}{r}\,\bm{\Psi}_{\ell m}.
\end{align}

\subsection*{Двойные роторы}

\paragraph{1. \(\displaystyle\nabla\times\nabla\times\!\bigl[A(r)\,\bm{Y}_{\ell m}\bigr]\).}

\begin{equation}
\nabla\times\nabla\times\!\bigl[A(r)\,\bm{Y}_{\ell m}\bigr]
  = -\frac{\lambda\,A(r)}{r^{2}}\;\bm{Y}_{\ell m}
    +\frac{A'(r)}{r}\;\bm{\Psi}_{\ell m}.
\end{equation}

\paragraph{2. \(\displaystyle\nabla\times\nabla\times\!\bigl[B(r)\,\bm{\Psi}_{\ell m}\bigr]\).}

\begin{equation}
\nabla\times\nabla\times\!\bigl[B(r)\,\bm{\Psi}_{\ell m}\bigr]
  = -\frac{\lambda}{r}\!\left(B'(r)+\frac{B(r)}{r}\right)\bm{Y}_{\ell m}
    -\left(B''(r)+\frac{2B'(r)}{r}\right)\bm{\Psi}_{\ell m}.
\end{equation}

\paragraph{3. \(\displaystyle\nabla\times\nabla\times\!\bigl[C(r)\,\bm{\Phi}_{\ell m}\bigr]\).}

\begin{equation}
\nabla\times\nabla\times\!\bigl[C(r)\,\bm{\Phi}_{\ell m}\bigr]
  =\left(\frac{\lambda\,C(r)}{r^{2}}
         -C''(r)-\frac{2C'(r)}{r}\right)\bm{\Phi}_{\ell m}.
\end{equation}

\subsection*{Проверка для диполя (\(\ell=1\))}

\[
\begin{array}{c|c}
C(r) & \bigl(\nabla\times\nabla\times[C(r)\bm{\Phi}_{1m}]\bigr) \\
\hline
\text{const} & \dfrac{2}{r^{2}}\bm{\Phi}_{1m} \\
r            & 0 \\
r^{2}        & -4\,\bm{\Phi}_{1m}
\end{array}
\]



\end{document}
