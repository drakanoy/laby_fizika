\documentclass{article}
\usepackage[utf8]{inputenc}
\usepackage[russian]{babel}
\usepackage{amsmath}
\usepackage{amssymb}
\usepackage{textcomp}


\begin{document}

\section*{Движение заряженной частицы в бесконечном соленоиде: общий случай с начальной скоростью \texorpdfstring{$(v_{x0},\,v_{y0})$}{(v\_x0, v\_y0)}}

Рассмотрим заряженную частицу (заряд $q$, масса $m$), влетающую через боковую поверхность бесконечного соленоида радиуса $R$ с однородным магнитным полем
\[
\mathbf{B} = (0,\,0,\,B),
\quad B>0
\]
внутри ($r<R$). Пусть момент влёта соответствует координатам
\[
(x_0,\,y_0) = (R,\,0),
\]
а компоненты начальной скорости в плоскости $xy$ равны
\[
\mathbf{v}_0 = (v_{x0},\,v_{y0}),
\quad
v_0 = \sqrt{v_{x0}^2 + v_{y0}^2}.
\]
Снаружи соленоида ($r>R$) поля нет.

\subsection*{1. Ларморов радиус и центр окружности}

Движение частицы в однородном магнитном поле по модулю скорости $v_0$ -- круговое (циклическое) в плоскости, перпендикулярной $\mathbf{B}$. Ларморов радиус
\[
\rho = \frac{m\,v_0}{|q|\,B}.
\]
Центр ларморовской орбиты $\bigl(x_c,\;y_c\bigr)$ (окружности радиуса $\rho$) в плоскости $xy$ находится по формуле:
\[
\mathbf{R}_c
=
\bigl(x_0,\,y_0\bigr)
+
\frac{m}{\,q\,B^2\,}
\bigl[\mathbf{v}_0 \times \mathbf{B}\bigr].
\]
Векторное произведение в координатах даёт:
\[
\mathbf{v}_0 = (v_{x0},\,v_{y0},\,0),
\quad
\mathbf{B} = (0,\,0,\,B),
\]
\[
\mathbf{v}_0 \times \mathbf{B}
=
B\,(v_{y0},\;-v_{x0},\,0).
\]
Следовательно:
\[
x_c
=
R + \frac{m\,v_{y0}}{\,q\,B\,},
\quad
y_c
=
-\frac{m\,v_{x0}}{\,q\,B\,}.
\]

\subsection*{2. Уравнение траектории}

Внутри соленоида ($r < R$) траектория частицы --- окружность радиуса $\rho$, заданная:
\[
(x - x_c)^2 + (y - y_c)^2 \;=\; \rho^2,
\]
причём $\rho = \dfrac{m}{\,|q|\,B\,}\,\sqrt{v_{x0}^2 + v_{y0}^2}$. Если заряд $q>0$ и $B>0$, вращение идёт \textit{по часовой} стрелке, поскольку сила Лоренца смещает вектор скорости вправо от радиуса.

\subsection*{3. Точка вылета \texorpdfstring{$(x_2,\,y_2)$}{(x2, y2)}}

Снаружи поля нет, значит при $r>R$ движение прямолинейно и равномерно. Предположим, что частица действительно дважды пересекает границу $r=R$ (т.\,е. \textbf{влетает} и \textbf{вылетает}):
\[
\begin{cases}
x^2 + y^2 = R^2,\\
(x - x_c)^2 + (y - y_c)^2 = \rho^2.
\end{cases}
\]
Эта система имеет два решения:
\[
(R,\,0) \quad (\text{точка влёта})
\quad\text{и}\quad
(x_2,\,y_2)\quad (\text{точка вылета}).
\]
Формально $(x_2,\,y_2)$ можно найти, вычитая уравнения окружностей:
\[
x^2 + y^2 = R^2,
\quad
(x - x_c)^2 + (y - y_c)^2 = \rho^2.
\]
Решение зачастую проще выполнить численно, чем выписывать в замкнутом виде.

\subsection*{4. Скорость при выходе: \texorpdfstring{$\mathbf{v}_{\text{выход}}$}{v\_exit}}

В однородном $\mathbf{B}$ вектор скорости равен
\[
\mathbf{v}(t)
=
-\,\Omega \,\Bigl[(\mathbf{r}(t) - \mathbf{R}_c)\times \hat{z}\Bigr],
\quad
\text{где }
\Omega = \frac{q\,B}{m} > 0
\]
(при $q>0,\,B>0$), а $\hat{z}=(0,0,1)$.

В момент вылета координаты частицы --- это $(x_2,\,y_2)$. Тогда:
\[
\mathbf{v}_{\text{выход}}
=
-\,\Omega
\bigl[\,(x_2 - x_c,\; y_2 - y_c,\,0)\,\times(0,0,1)\bigr].
\]
Вспомним, что векторное произведение $(X,\,Y,\,0)\times(0,0,1)$ даёт $(\,Y,\;-X,\,0)$. Значит:
\[
\mathbf{v}_{\text{выход}}
=
\Bigl(-\,\Omega\,(y_2 - y_c),\;\;\Omega\,(x_2 - x_c),\;0\Bigr).
\]
Поскольку
\(
\Omega\,\rho = \dfrac{q\,B}{m}\,\dfrac{m\,v_0}{|q|\,B} = v_0,
\)
модуль скорости остаётся $v_0$, и частица выходит из соленоида, сохранив ту же кинетическую энергию.

Итоговые компоненты на выходе:
\[
v_{x,\text{выход}}
=
-\,\frac{q\,B}{m}\,\bigl(y_2 - y_c\bigr),
\quad
v_{y,\text{выход}}
=
\frac{q\,B}{m}\,\bigl(x_2 - x_c\bigr).
\]
\[
|\mathbf{v}_{\text{выход}}| = v_0.
\]

\subsection*{5. Выводы}

\begin{enumerate}
  \item \textbf{Модуль скорости не меняется} при движении внутри соленоида, так как магнитная сила не совершает работы.
  \item \textbf{Направление} скорости на выходе отличается от начального и определяется тем, \emph{какую дугу} ларморовской окружности проходит частица внутри поля.
  \item Для явного вычисления $(v_{x,\text{выход}},\,v_{y,\text{выход}})$ в общем случае:
  \begin{itemize}
    \item Находим центр орбиты $(x_c,y_c)$;
    \item Решаем систему \textit{(окружность радиуса $R$)} $\cap$ \textit{(окружность ларморовской траектории радиуса $\rho$)} для определения второй точки пересечения $(x_2,y_2)$;
    \item Подставляем $(x_2,y_2)$ в формулу для скорости.
  \end{itemize}
\end{enumerate}

\vspace{0.5cm}
\noindent
\textbf{Примечание.} Если $\rho < R$, тогда траектория целиком умещается внутри $r<R$, и частица не вылетает через боковую поверхность. Если $\rho = R$ и нет радиальной компоненты скорости, возможно касательное пересечение без повторного пересечения границы.
%%%%%%%%%%%%%%%%%%%%%%%%%%%%%%%%%%%%%%%%%%%%%%%%%%%%%%%%%%%%%%%%%%%%%%%%%%%%%%%%%%%%%%%%%%%
\section*{Выражение для скорости заряженной частицы в однородном магнитном поле}

Рассмотрим уравнение движения частицы с зарядом $q$ и массой $m$ в однородном магнитном поле
\[
\mathbf{B} = (0,\,0,\,B),
\quad B>0.
\]
Отсутствуют электрическое поле и силы трения (резистивные силы).

\subsection*{1. Уравнение движения и разбиение по компонентам}

Сила Лоренца:
\[
\mathbf{F} = q \,\bigl[\mathbf{v}\times \mathbf{B}\bigr].
\]
По второму закону Ньютона:
\[
m\,\frac{d\mathbf{v}}{dt}
=
q\,\bigl[\mathbf{v}\times \mathbf{B}\bigr].
\]
Разложим скорость
\(\mathbf{v} = (v_x,\,v_y,\,v_z)\)
и поле
\(\mathbf{B} = (0,\,0,\,B)\).
Тогда
\[
[\mathbf{v}\times \mathbf{B}]
=
(v_x,\,v_y,\,v_z)\times (0,\,0,\,B)
=
\bigl(B\,v_y,\;-B\,v_x,\,0\bigr).
\]
Получаем систему:
\[
\begin{cases}
m\,\dfrac{dv_x}{dt} = q\,B\,v_y,\\
m\,\dfrac{dv_y}{dt} = -\,q\,B\,v_x,\\
m\,\dfrac{dv_z}{dt} = 0.
\end{cases}
\]

\subsection*{2. Решение для \texorpdfstring{$v_z(t)$}{vz(t)}}

Из
\[
m \,\frac{dv_z}{dt} = 0
\quad\Longrightarrow\quad
\frac{dv_z}{dt} = 0
\quad\Longrightarrow\quad
v_z(t)= \mathrm{const} = v_{z0}.
\]
Таким образом, проекция скорости вдоль $B$ не меняется.

\subsection*{3. Решение для \texorpdfstring{$v_x(t)$}{vx(t)} и $v_y(t)$}

Обозначим
\[
\Omega = \frac{q\,B}{m}.
\]
Тогда уравнения для $(v_x,\,v_y)$ примут вид:
\[
\begin{cases}
\dfrac{dv_x}{dt} = \Omega\,v_y,\\
\dfrac{dv_y}{dt} = -\,\Omega\,v_x.
\end{cases}
\]
Это система гармонических колебаний с частотой~$\Omega$. Дифференцируя первое уравнение ещё раз, получаем:
\[
\frac{d^2 v_x}{dt^2}
=
\Omega\,\frac{dv_y}{dt}
=
\Omega \,\bigl(-\,\Omega\,v_x\bigr)
=
-\,\Omega^2 \,v_x.
\]
Значит
\[
\frac{d^2 v_x}{dt^2} + \Omega^2\,v_x = 0.
\]
Общее решение:
\[
v_x(t)
=
A\,\cos(\Omega t) + B\,\sin(\Omega t).
\]
Используя
\(\dfrac{dv_x}{dt} = \Omega\,v_y\),
получаем:
\[
v_y(t)
=
\frac{1}{\Omega}\,\frac{dv_x}{dt}
=
-\,A\,\sin(\Omega t) + B\,\cos(\Omega t).
\]

\subsection*{4. Учёт начальных условий}

Пусть
\[
\mathbf{v}(0) = (v_{x0},\,v_{y0},\,v_{z0}).
\]
Тогда в момент $t=0$:
\[
v_x(0)
=
A
=
v_{x0},
\quad
v_y(0)
=
B
=
v_{y0}.
\]
Следовательно, $A=v_{x0}$, $B=v_{y0}$. Подставляя назад:
\[
\begin{aligned}
v_x(t)
&=
v_{x0}\,\cos(\Omega t) \;+\; v_{y0}\,\sin(\Omega t),\\
v_y(t)
&=
-\,v_{x0}\,\sin(\Omega t) \;+\; v_{y0}\,\cos(\Omega t),
\end{aligned}
\quad
v_z(t)
=
v_{z0}.
\]
Таким образом,
\[
\boxed{
\mathbf{v}(t)
=
\bigl(v_x(t),\,v_y(t),\,v_z(t)\bigr)
=
\Bigl(\,
v_{x0}\,\cos(\Omega t) + v_{y0}\,\sin(\Omega t),\;
-\,v_{x0}\,\sin(\Omega t) + v_{y0}\,\cos(\Omega t),\;
v_{z0}
\Bigr).
}
\]
Здесь
\(\Omega = \tfrac{q\,B}{m}\)
может быть положительным или отрицательным, что определяет направление вращения. Если $q>0$, $B>0$, вращение идёт \emph{по часовой} стрелке, смотря вдоль оси $z$.

\subsection*{5. Физическая картина}

\begin{itemize}
  \item Если $v_{z0} = 0$, движение в плоскости, перпендикулярной $B$, идёт \textbf{по окружности} с угловой скоростью $\Omega$ (частота циклотронной вращения).
  \item При $v_{z0}\neq 0$ траектория --- \textbf{винтовая линия} (спираль) вокруг оси $z$.
  \item Модуль скорости сохраняется, так как сила Лоренца перпендикулярна $\mathbf{v}$ и не совершает работы.
\end{itemize}

%%%%%%%%%%%%%%%%%%%%%%%%%%%%%%%%%%%%%%%%%%%%%%%%%%%%%%%%%%%%%

\section*{Почему возникает формула
\(\displaystyle \mathbf{v}(t)
=
-\,\Omega \,\bigl[(\mathbf{r}(t) - \mathbf{R}_c)\times \hat{z}\bigr]\)?}

Рассмотрим заряженную частицу (с зарядом \(q>0\) и массой \(m\)) в однородном магнитном поле
\(\mathbf{B}=(0,0,B)\) с \(B>0\). В плоскости, перпендикулярной \(\mathbf{B}\), частица движется по \textbf{окружности} с радиусом \(\rho\) (ларморов радиус), совершая вращение \emph{по часовой стрелке}, если смотреть вдоль \(+z\).

Ниже объясняется, почему вектор скорости \(\mathbf{v}(t)\) можно записать как
\[
\mathbf{v}(t)
=
-\,\Omega \,\Bigl[(\mathbf{r}(t) - \mathbf{R}_c)\times \hat{z}\Bigr],
\quad
\text{где }
\Omega = \frac{q\,B}{m} > 0.
\]
Знак <<\(-\)>> отражает именно направление вращения (по часовой стрелке).

\subsection*{1. Центр окружности и радиус-вектор}

\begin{itemize}
  \item Вектор \(\mathbf{R}_c\) --- центр ларморовой окружности, вокруг которого происходит движение.
  \item \(\mathbf{r}(t) - \mathbf{R}_c\) --- это радиус-вектор от центра окружности до текущего положения частицы.
\end{itemize}

Для однородного магнитного поля \(\mathbf{B}\), расположенного вдоль \(z\), известно, что
\[
\mathbf{R}_c
=
\mathbf{r}_0
+
\frac{m}{\,q\,B^2\,}\,\bigl[\mathbf{v}_0 \times \mathbf{B}\bigr]
\]
(в случае, если \(\mathbf{r}_0\) и \(\mathbf{v}_0\) --- начальные положение и скорость).

\subsection*{2. Круговое движение и поворот на \(90^\circ\)}

Если частица движется по окружности радиуса \(\rho\) с угловой скоростью \(\Omega = \dfrac{q\,B}{m}\), то вектор \(\mathbf{r}(t) - \mathbf{R}_c\) постоянно поворачивается \emph{по часовой стрелке} в плоскости \(xy\). Модуль скорости при таком движении равен \(\Omega\,\rho\).

\[
\text{Пусть } \mathbf{r}(t) = \mathbf{R}_c + \boldsymbol{\rho}(t),
\]
где \(\boldsymbol{\rho}(t)\) --- вектор длины \(\rho\), вращающийся с частотой \(\Omega\). Время\-вая производная даёт скорость:
\[
\mathbf{v}(t)
=
\frac{d\mathbf{r}(t)}{dt}
=
\frac{d\boldsymbol{\rho}(t)}{dt}.
\]
Вращение на \(90^\circ\) \emph{вправо} (по часовой стрелке) в векторной форме выражается операцией \(-\,[\dots \times \hat{z}]\). Именно это и даёт минус в формуле.

\subsection*{3. Вывод формулы \(\mathbf{v}(t)
=
-\,\Omega\,[(\mathbf{r}(t)-\mathbf{R}_c)\times \hat{z}]\)}

Пусть \(\mathbf{r}(t) - \mathbf{R}_c = \boldsymbol{\rho}(t)\). При круговом движении:
\[
\frac{d\boldsymbol{\rho}(t)}{dt}
=
-\,\Omega\,\bigl[\boldsymbol{\rho}(t)\times \hat{z}\bigr],
\]
поскольку нужно повернуть \(\boldsymbol{\rho}(t)\) на \(90^\circ\) вправо и умножить её длину на \(\Omega\). Следовательно,
\[
\mathbf{v}(t)
=
\frac{d\boldsymbol{\rho}(t)}{dt}
=
-\,\Omega\,\bigl[(\mathbf{r}(t) - \mathbf{R}_c)\times \hat{z}\bigr].
\]
Так получаем искомую формулу.

\subsection*{4. Роль знака <<\(-\)>>}

\begin{itemize}
  \item Операция \(\mathbf{a}\times \hat{z}\) обычно соответствует повороту \(\mathbf{a}\) \emph{против} часовой стрелки (если смотреть сверху вдоль \(+z\)).
  \item Здесь же вращение идёт по часовой стрелке (при \(q>0,\,B>0\)). Чтобы учесть <<обратный>> поворот, появляется дополнительный знак <<\(-\)>>.
\end{itemize}

\subsection*{Итог}

Таким образом, при движении заряженной частицы по окружности в однородном магнитном поле вдоль \(z\) можно выписать скорость в любой момент времени \(t\) так:
\[
\boxed{
\mathbf{v}(t)
=
-\,\Omega \,\Bigl[(\mathbf{r}(t) - \mathbf{R}_c)\times \hat{z}\Bigr],
\quad
\Omega = \frac{q\,B}{m}>0,
}
\]
где \(\mathbf{R}_c\) есть центр ларморовской орбиты. Знак <<\(-\)>> соответствует часовой ориентации вращения в плоскости \(xy\) при \(q>0\), \(B>0\).


%%%%%%%%%%%%%%%%%%%%%%%%%%%%%%%%%%%%%%%%%%%%%%%%%%%%%%%%%%%%%%%%%%%%
\end{document}