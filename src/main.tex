%! Author = drakanoy
%! Date = 22.12.2023

% Preamble
\documentclass[12pt]{article}

% Packages
\usepackage[utf8]{inputenc}
\usepackage[T2A]{fontenc}
\usepackage[english, russian]{babel}
\usepackage[a4paper, includefoot, left=1.5cm, right=1.5cm, top=1cm, bottom=1.5cm, headsep=1cm, footskip=1cm]{geometry}
\usepackage{makecell}
\usepackage{amsmath}
\usepackage{graphicx}
\usepackage{enumitem}
\usepackage{svg}
\usepackage{multirow}
\usepackage{hyperref}
\usepackage{mathtools}

% Document
\begin{document}
\begin{large}
\begin{center}
\LARGE \textbf{Преобразования Лоренца}
\end{center}
\par В физике преобразования Лоренца - линейная группа преобразований пространства времени,
относительно которой законы физики должны быть инварианты.
\par А именно, если $A_1(\vec{r}; ct); \dots; A_n(\vec{r}; ct)$ - физические величины,
\newline $F(A_1; \dots, A_n) = 0$ - закон, связывающий их, то после преобразований $\Lambda$:
    \begin{equation}
        \begin{pmatrix}
            ct'\\
            x'\\
            y'\\
            z'
        \end{pmatrix}
        = \Lambda \cdot
        \begin{pmatrix}
            ct\\
            x\\
            y\\
            z
        \end{pmatrix}
    \end{equation}
Новые величины $A'_1(\vec{r'}; ct'); \dots; A'_n(\vec{r'}; ct')$ будут подчиняться тому же закону
\newline $F(A'_1; \dots, A'_n) = 0$.
    \par Группа Лоренца состоит из 3-х групп: поворотов, сдвигов и Лоренцевых бустов.
    Про первые 2 все понятно. Мы знаем как выглядит общая матрица поворота и сдвиг:
    \begin{equation}
        \begin{pmatrix}
            cos(\theta) & -sin(\theta)\\
            sin(\theta) & cos(\theta)
        \end{pmatrix};
        \begin{pmatrix}
            ct'\\
            x'\\
            y'\\
            z'
        \end{pmatrix}
        =
        \begin{pmatrix}
            ct_0\\
            x_0\\
            y_0\\
            z_0
        \end{pmatrix}
        +
        \begin{pmatrix}
            ct\\
            x\\
            y\\
            z
        \end{pmatrix}
    \end{equation}
    Я получу матрицу Лоренцево буста в общем виде.
    \par Всем достаточно хорошо известна эта матрица в случае сонаправленности скорости с одной из осей (например х):
    \begin{equation}
        \Lambda^i_{\text{ }j} =
        \begin{pmatrix}
            \frac{1}{\sqrt{1-\frac{v^2}{c^2}}} & \frac{-\frac{v}{c}}{\sqrt{1-\frac{v^2}{c^2}}} & 0 & 0 \\
            \frac{-\frac{v}{c}}{\sqrt{1-\frac{v^2}{c^2}}} & \frac{1}{\sqrt{1-\frac{v^2}{c^2}}} & 0 & 0 \\
            0 & 0 & 1 & 0 \\
            0 & 0 & 0 & 1
        \end{pmatrix}
    \end{equation}
    А если обозначить $\beta \coloneqq \frac{v}{c}$; $\gamma \coloneqq \frac{1}{\sqrt{1 - \beta}}$ то:
    \begin{equation}
        \Lambda^i_{\text{ }j} =
        \begin{pmatrix}
            \gamma & -\beta \gamma & 0 & 0 \\
            -\beta \gamma & \gamma & 0 & 0 \\
            0 & 0 & 1 & 0 \\
            0 & 0 & 0 & 1
        \end{pmatrix}
    \end{equation}
    Но что делать если нужно совершить буст в направлении не совпадающим с осями, например когда $\vec{v} = (v_x; v_y; v_z)$?
    Из 1-ого случая мы видим что какое-либо изменение происходит по оси, параллельной направлению скорости.
    \par Распишем:
    \begin{equation}
    \begin{gathered}
        \vec{r} = \vec{r_{\parallel}} + \vec{r_\perp} \\
        \vec{r}\text{ }' = \vec{r_{\parallel}}' + \vec{r_\perp}'
    \end{gathered}
    \end{equation}
    \par Где:
    \begin{equation}
    \begin{gathered}
        \vec{r_{\parallel}} = \left(\vec{r} \cdot \frac{\vec{v}}{v}\right) \frac{\vec{v}}{v} \\
        \vec{r_\perp} = \vec{r} - \left(\vec{r} \cdot \frac{\vec{v}}{v}\right) \frac{\vec{v}}{v}
    \end{gathered}
    \end{equation}
    \par Тогда:
    \begin{equation}
    \begin{gathered}
        ct' = \gamma \left( ct - \left(\vec{\beta} \cdot \vec{r_{\parallel}}\right)   \right) \\
        \vec{r_{\parallel}}' = \gamma \left(\vec{r_\parallel} - \vec{\beta}ct \right) \\
        \vec{r_\perp}' = \vec{r_\perp}
    \end{gathered}
    \end{equation}
    \par В общем виде:
    \begin{equation}
    \begin{gathered}
        ct' = \gamma \left( ct - \beta \left(\vec{r} \cdot \frac{\vec{v}}{v}\right)  \right) \\
        \vec{r}\text{ }' = \vec{r} + \left( \gamma - 1 \right)\left(\vec{r} \cdot \frac{\vec{v}}{v}\right) \frac{\vec{v}}{v} - \gamma\vec{\beta}ct
    \end{gathered}
    \end{equation}
    \par И матрица преобразования Лоренца в общем случае:
    \begin{equation}
        \Lambda^i_{\text{ }j} =
        \begin{pmatrix}
            \gamma & -\gamma\frac{v_x}{c} & -\gamma\frac{v_y}{c} & -\gamma\frac{v_z}{c}\\
            -\gamma\frac{v_x}{c} & 1 + \left( \gamma - 1 \right)\frac{v_x^2}{v^2} & \left( \gamma - 1 \right)\frac{v_x v_y}{v^2} & \left( \gamma - 1 \right)\frac{v_x v_z}{v^2} \\
            -\gamma\frac{v_y}{c} & \left( \gamma - 1 \right)\frac{v_y v_x}{v^2} & 1 + \left( \gamma - 1 \right)\frac{v_y^2}{v^2} & \left( \gamma - 1 \right)\frac{v_y v_z}{v^2} \\
            -\gamma\frac{v_z}{c} & \left( \gamma - 1 \right)\frac{v_z v_x}{v^2} & \left( \gamma - 1 \right)\frac{v_z v_y}{v^2} & 1 + \left( \gamma - 1 \right)\frac{v_z^2}{v^2}
        \end{pmatrix}
    \end{equation}
    \par Для скорости тоже общий вид напишу для красоты, получить который можно расписав дифференциалы:
    \begin{equation}
    \begin{gathered}
        \vec{u} = \frac{d\vec{r}}{dt} \\
        \vec{u}\text{ }' = \frac{d\vec{r}\text{ }'}{dt} \\
        \gamma_v \coloneqq \frac{1}{\sqrt{1 - \frac{v^2}{c^2}}} \\
        \vec{u}\text{ }' = \frac{1}{1 - \frac{\vec{v} \cdot \vec{u}}{c^2}}\left( \frac{\vec{u}}{\gamma_v} -\vec{v} + \frac{1}{c^2}\frac{\gamma_v}{1 + \gamma_v}\left( \vec{u} \cdot \vec{v} \right) \cdot \vec{v} \right)
    \end{gathered}
    \end{equation}
    \par Ну и на десерт формула, которую я мало понимаю, но выглядящую красиво. Преобразование общего \textsl{невзаимодействующего} многочастичного состояния (состояния из пространства Фока) в квантовой теории поля:
    \begin{equation}
    \begin{normalsize}
    \begin{gathered}
        U(\Lambda; a)\Psi_{p_1\sigma_1 n_1;p_2\sigma_2 n_2;\dots} = \\
        = e^{-ia_\mu\left[ \left( \Lambda p_1 \right)^\mu + \left( \Lambda p_2 \right)^\mu + \dots \right]}\sqrt{\frac{\left( \Lambda p_1 \right)^0 \left( \Lambda p_2 \right)^0 \dots}{p_1^0 p_2^0 \dots}}\left( \sum\limits_{\sigma'_1 \sigma'_2 \dots} D_{\sigma'_1 \sigma_1}^{(j_1)}[W(\Lambda; p_1)] D_{\sigma'_2 \sigma_2}^{(j_2)}[W(\Lambda; p_2)] \dots\right)\Psi_{\Lambda p_1\sigma'_1 n_1;\Lambda p_2\sigma'_2 n_2;\dots}
    \end{gathered}
    \end{normalsize}
    \end{equation}
    \par Где $W(\Lambda; p)$ - вигнеровское вращение, a $D^{(j)}$ - $(2 j + 1)$-мерное представление группы $SO(3)$.
\end{large}
\end{document}
