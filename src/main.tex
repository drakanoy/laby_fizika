\documentclass{article}
\usepackage[utf8]{inputenc}
\usepackage[russian]{babel}
\usepackage{amsmath}
\usepackage{amssymb}

\begin{document}

Беседа № 30:

Решим эту задачу аналитически, рассчитывая распределение случайной величины \( N \). Напомню, что \( N = \sup \{ n : U_1 U_2 \cdots U_n > s \} \), где \( U_i \) — независимые равномерно распределенные случайные величины на отрезке \([0, 1]\).

1. Определение зависимости:

Мы можем переписать условие:

\[
U_1 U_2 \cdots U_n > s
\]

или в логарифмическом виде:

\[
\log(U_1) + \log(U_2) + \cdots + \log(U_n) > \log(s).
\]

Логарифм равномерно распределенной случайной величины на \([0, 1]\) имеет распределение с плотностью:

\[
f_{\log(U)}(x) = e^x \quad \text{для } x < 0,
\]

так как \( U \) принимает значения от 0 до 1.

Определим новую случайную величину \( X_i = -\log(U_i) \). Тогда \( X_i \) имеет экспоненциальное распределение с параметром \( \lambda = 1 \) (среднее значение \( 1 \)).

2. Сумма случайных величин:

Теперь мы можем заключить, что:
\[
Y_n = -(\log(U_1) + \log(U_2) + \cdots + \log(U_n)) = X_1 + X_2 + \cdots + X_n.
\]

Сумма \( Y_n \) имеет распределение Гамма с параметрами \( n \) и 1:

\[
Y_n \sim \text{Гамма}(n, 1),
\]

что эквивалентно:

\[
Y_n \sim \text{Сумма } n \text{ независимых } \text{exp}(1).
\]

3. Оценка вероятности:

Таким образом, мы хотим оценить:

\[
P(N \geq n) = P(U_1 U_2 \cdots U_n > s) = P(Y_n < -\log(s)).
\]

Плотность Гамма \( Y_n \) задается как:

\[
f_Y(y; n, 1) = \frac{y^{n-1} e^{-y}}{(n-1)!} \quad \text{для } y > 0.
\]

4. Вероятность:

Чтобы найти вероятности, нам нужно вычислить:

\[
P(Y_n < -\log(s)) = \int_0^{-\log(s)} \frac{y^{n-1} e^{-y}}{(n-1)!} dy,
\]

что является функцией распределения Гамма и может быть связано с функцией \( \Gamma \).

 5. Использование функции распределения Гамма:

Таким образом,

\[
P(N \geq n) = 1 - P(Y_n \geq -\log(s)) = 1 - \int_{-\log(s)}^{\infty} \frac{y^{n-1} e^{-y}}{(n-1)!} dy = 1 - \Gamma(n, -\log(s)),
\]

где \( \Gamma(n, x) \) — это неполная гамма-функция.

6. Формула для \( P(N = n) \):

Теперь, чтобы найти \( P(N = n) \),

\[
P(N = n) = P(N \geq n) - P(N \geq n+1).
\]

Подставив значения, получим:

\[
P(N = n) = \left(1 - \Gamma(n, -\log(s))\right) - \left(1 - \Gamma(n+1, -\log(s))\right)
\]
\[
= \Gamma(n+1, -\log(s)) - \Gamma(n, -\log(s)).
\]
Заключение:

Таким образом, распределение \( N \) можно найти через используя неполные гамма-функции:

\[
P(N = n) = \Gamma(n+1, -\log(s)) - \Gamma(n, -\log(s)),
\]
что завершает аналитическое решение задачи.

Примечание:

- Гамма-функция и её неполные версии часто доступны в математических библиотеках, таких как SciPy в Python, что позволяет легко подсчитывать вероятность \( P(N = n) \) для конкретных значений \( s \)

\end{document}