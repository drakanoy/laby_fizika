\documentclass{article}
\usepackage[utf8]{inputenc}
\usepackage[russian]{babel}
\usepackage{amsmath}
\usepackage{amssymb}
\usepackage{textcomp}

\begin{document}

\[
\textbf{Рассмотрим трапецию с вершинами }(0,0),\,(0,2),\,(2,1)\text{ и }(2,0).
\]
\[
\text{Пусть точка }(X,Y)\text{ равномерно распределена по периметру этой трапеции.}
\]
\[
\text{Периметр состоит из отрезков:}
\]
\[
AB: (0,0)\to (0,2),\quad L_1 = 2;
\quad BC: (0,2)\to (2,1),\quad L_2 = \sqrt{(2-0)^2 + (1-2)^2}=\sqrt{5};
\]
\[
CD: (2,1)\to(2,0),\quad L_3=1;
\quad DA: (2,0)\to(0,0),\quad L_4=2.
\]
\[
\text{Суммарная длина периметра }L = 2 + \sqrt{5} + 1 + 2 = 5 + \sqrt{5}.
\]

\[
\text{Для }0 < X < 2\text{ точка может лежать либо на }BC\text{, либо на }DA.
\]
\[
\text{На }BC\text{ имеем }(x,y)=(2t,\,2-t),\;t\in[0,1].\;
ds=\sqrt{5}\,dt,\;dx=2\,dt.
\]
\[
\text{На }DA\text{ имеем }(x,y)=(2s,0),\;s\in[0,1].\;
ds=2\,ds_x \text{ (или просто }dx\text{), где }y=0.
\]

\[
\text{Отношение вероятностей «попасть» на }BC\text{ к }DA
= \frac{\sqrt{5}}{1}.
\]
\[
P(\text{на }BC\mid X=x)
= \frac{\sqrt{5}}{\sqrt{5}+1},\quad
P(\text{на }DA\mid X=x)
= \frac{1}{\sqrt{5}+1}.
\]
\[
\text{Следовательно, при }0<x<2:
\]
\[
E[Y\mid X=x]
= \frac{\sqrt{5}}{\sqrt{5}+1}\,\Bigl(2 - \frac{x}{2}\Bigr).
\]

\[
\text{На границах }x=0 \text{ и } x=2:
\]
\[
x=0\colon (0,0)\to(0,2)\;\Longrightarrow\;Y\in[0,2],
\quad E[Y\mid X=0] = 1.
\]
\[
x=2\colon (2,1)\to(2,0)\;\Longrightarrow\;Y\in[0,1],
\quad E[Y\mid X=2] = 0.5.
\]

\[
\boxed{
E[Y\mid X=x] =
\begin{cases}
1, & x=0,\\[6pt]
\dfrac{\sqrt{5}}{\sqrt{5}+1}\Bigl(2 - \tfrac{x}{2}\Bigr), & 0<x<2,\\[6pt]
0.5, & x=2.
\end{cases}
}
\]

\[
\text{Например, при }x=1\text{ получаем }
E[Y\mid X=1]
= \frac{\sqrt{5}}{\sqrt{5}+1}\left(2 - \tfrac12\right)
= \frac{3\sqrt{5}}{2(\sqrt{5}+1)} \approx 1.035.
\]
\end{document}