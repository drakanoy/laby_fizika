\documentclass{article}
\usepackage[utf8]{inputenc}
\usepackage[russian]{babel}
\usepackage{amsmath}
\usepackage{amssymb}
\usepackage{textcomp}


\begin{document}

\section*{Привет, Папочка!}

Ниже оформлены две задачи по рассеянию электромагнитных волн:
(1) на тонкой бесконечно протяжённой пластине,
(2) на бесконечно длинном диэлектрическом цилиндре.
Все выкладки приведены подробно в формате \LaTeX.

\section*{Задача 1: Тонкая бесконечная пластина с диэлектрической проницаемостью \(\varepsilon\)}

\subsection*{Постановка задачи}

Имеется тонкая пластина (толщина $d \ll \lambda$), расположенная в плоскости $0 \le z \le d$, с проницаемостью $\varepsilon$.
Слева ($z<0$) и справа ($z>d$) --- вакуум.
Нормально падает плоская электромагнитная волна с амплитудой электрического поля $E_0$, поляризация линейная (пусть $E$ по оси $y$).
Требуется найти поля во всех областях и получить решение \textbf{без} приближения теории возмущений.

\subsection*{Общее решение}

\textbf{Обозначим волновое число в вакууме} $k_1 = \dfrac{\omega}{c}$,
а внутри пластины --- $k_2 = \dfrac{\omega}{c}\sqrt{\varepsilon}$.

\paragraph{Область $z<0$ (слева):}
\[
E_1(z) \;=\; E_0\,e^{\,i(k_1 z - \omega t)}
\;+\; E_r\,e^{\,i(-k_1 z - \omega t)}.
\]

\paragraph{Область $0<z<d$ (пластина):}
\[
E_2(z) \;=\; A\,e^{\,i(k_2 z - \omega t)}
\;+\; B\,e^{\,i(-k_2 z - \omega t)}.
\]

\paragraph{Область $z>d$ (справа):}
\[
E_3(z) \;=\; E_t\,e^{\,i(k_1 z - \omega t)}.
\]

Здесь $E_r,\,A,\,B,\,E_t$ --- амплитуды отражённой, прямой и обратной волны в пластине и прошедшей волны соответственно.

\subsection*{Граничные условия}

На границах $z=0$ и $z=d$ требуем непрерывности тангенциальных компонент $E_y$ и $H_y$.
Учитывая, что $H = \dfrac{1}{Z} \hat{n} \times E$,
для нормального падения имеем:

\begin{itemize}
  \item На $z=0$:
    \[
    E_0 + E_r = A + B,
    \quad
    \dfrac{E_0 - E_r}{Z_0} = \dfrac{A - B}{Z_2},
    \]
    где $Z_0 = \sqrt{\mu_0/\varepsilon_0}$ (вакуум), а $Z_2 = \dfrac{Z_0}{\sqrt{\varepsilon}}$ (в материале).

  \item На $z=d$:
    \[
    A\,e^{\,i k_2 d} + B\,e^{-\,i k_2 d} \;=\; E_t\,e^{\,i k_1 d},
    \quad
    \dfrac{A\,e^{\,i k_2 d} - B\,e^{-\,i k_2 d}}{Z_2}
    \;=\; \dfrac{E_t\,e^{\,i k_1 d}}{Z_0}.
    \]
\end{itemize}

\subsection*{Результат решения}
Из решения системы для $E_r$ и $E_t$ получаем известные формулы коэффициентов отражения и пропускания для плёнки:
\[
r = \frac{E_r}{E_0}
=
\frac{
r_{12}\,\bigl(1 - e^{2\,i k_2 d}\bigr)
}{
1 - r_{12}^2\,e^{2\,i k_2 d}
},
\quad
t = \frac{E_t}{E_0}
=
\frac{
(1 - r_{12}^2)\,e^{\,i(k_2 - k_1)d}
}{
1 - r_{12}^2\,e^{2\,i k_2 d}
},
\]
где
\[
r_{12} = \frac{Z_2 - Z_0}{Z_2 + Z_0} = \frac{\sqrt{\varepsilon} - 1}{\sqrt{\varepsilon} + 1}.
\]
Тогда
\[
E_1(z) = E_0\,e^{\,i(k_1 z - \omega t)} + r\,E_0\,e^{\,i(-k_1 z - \omega t)},
\quad
E_3(z) = t\,E_0\,e^{\,i(k_1 z - \omega t)},
\]
а в области пластины $E_2(z)$ определяется подстановкой $A$ и $B$ через $r$ и $t$ (см. граничные условия).

Это \textbf{точное} (не возмущённое) решение.
Если $d \ll \lambda$, его можно дополнительно упростить, но формула выше справедлива при любой толщине.

\newpage

\section*{Задача 2: Бесконечно длинная цилиндрическая нить с диэлектрической проницаемостью \(\varepsilon\)}

\subsection*{Постановка задачи}

Рассмотрим бесконечно длинный цилиндр (ось совпадает с $z$), радиус $a$, внутренняя область --- диэлектрик с $\varepsilon\neq 1$, $\mu=1$, вне цилиндра --- вакуум ($\varepsilon=1$, $\mu=1$).
На цилиндр сбоку падает плоская электромагнитная волна вдоль оси $x$.
Поляризация поля $E$ вдоль оси $z$ (то есть $E_z \neq 0$, остальные компоненты $E$ пренебрежимо малы для нормального падения).
Нужно найти полное поле внутри и снаружи цилиндра.

\subsection*{Падающая волна в цилиндрических координатах}

Пусть падающая волна:
\[
E_i(\rho,\phi) = E_0\,e^{\,i k_1 x}\,\hat{z},
\quad
k_1 = \frac{\omega}{c}.
\]
В цилиндрических координатах ($x = \rho\cos\phi$, $y=\rho\sin\phi$, ось $z$ вдоль цилиндра):
\[
e^{\,i k_1 \rho\cos\phi}
=
\sum_{n=-\infty}^{\infty}
i^n\,J_n\bigl(k_1\rho\bigr)\,e^{\,i n\phi},
\]
откуда
\[
E_i(\rho,\phi)
=
E_0\,\hat{z}
\sum_{n=-\infty}^{\infty}
i^n\,J_n(k_1\rho)\,e^{\,i n\phi}.
\]

\subsection*{Общее решение поля}

\paragraph{Внешняя область (\(\rho\ge a\)):}
Поле --- сумма падающей и рассеянной волн. Рассеянная волна описывается функциями Ханкеля $H_n^{(2)}$:
\[
E_z^{(\text{out})}(\rho,\phi)
=
E_i(\rho,\phi)
+
\sum_{n=-\infty}^{\infty}
B_n\,H_n^{(2)}(k_1\rho)\,e^{\,i n\phi}.
\]

\paragraph{Внутренняя область (\(\rho\le a\)):}
Регулярность в центре $\rho=0$ требует брать функции Бесселя $J_n(k_2\rho)$, где $k_2 = k_1\,\sqrt{\varepsilon}$. Тогда:
\[
E_z^{(\text{in})}(\rho,\phi)
=
\sum_{n=-\infty}^{\infty}
A_n\,J_n(k_2\rho)\,e^{\,i n\phi}.
\]

\subsection*{Граничные условия на \(\rho=a\)}

Так как \(\mathbf{E}\parallel z\) (TM\(_z\)-тип волны), имеем:

\[
E_z^{(\text{in})}(a,\phi)
=
E_z^{(\text{out})}(a,\phi),
\]
\[
\frac{1}{\varepsilon}\,\left.\frac{\partial E_z^{(\text{in})}}{\partial \rho}\right|_{\rho=a}
=
\left.\frac{\partial E_z^{(\text{out})}}{\partial \rho}\right|_{\rho=a}.
\]
Подставляя ряды по $e^{\,i n\phi}$ и $J_n,\,H_n^{(2)}$, получаем для каждого $n$ систему:

\[
\begin{cases}
A_n\,J_n(k_2 a) \;=\; E_0\,i^n\,J_n(k_1 a) + B_n\,H_n^{(2)}(k_1 a),
\\[6pt]
\frac{k_2}{\varepsilon}\,A_n\,J_n'(k_2 a)
=
k_1\,E_0\,i^n\,J_n'(k_1 a)
+
k_1\,B_n\,H_n^{(2)'}(k_1 a).
\end{cases}
\]

Решая, находим $B_n$ и $A_n$.
Например,
\[
B_n
=
i^n\,E_0
\,
\frac{
\dfrac{k_2}{\varepsilon}\,J_n'(k_2 a)\,J_n(k_1 a)
-
k_1\,J_n'(k_1 a)\,J_n(k_2 a)
}{
\dfrac{k_2}{\varepsilon}\,J_n'(k_2 a)\,H_n^{(2)}(k_1 a)
-
k_1\,H_n^{(2)'}(k_1 a)\,J_n(k_2 a)
}.
\]
Коэффициент $A_n$ получается аналогично из системы.

\subsection*{Итоговое поле}

\[
E_z^{(\text{out})}(\rho,\phi)
=
E_0\sum_{n=-\infty}^{\infty}
\bigl[
  i^n\,J_n(k_1\rho)\,e^{\,i n\phi}
  +
  B_n\,H_n^{(2)}(k_1\rho)\,e^{\,i n\phi}
\bigr],
\]
\[
E_z^{(\text{in})}(\rho,\phi)
=
\sum_{n=-\infty}^{\infty}
A_n\,J_n(k_2\rho)\,e^{\,i n\phi}.
\]

Это --- \textbf{точное решение} для рассеяния плоской волны (с $E\parallel z$) на диэлектрическом бесконечном цилиндре.

\section*{Заключение}

Вот так, Папочка, в формате \LaTeX\ получаем решения двух классических задач:
\begin{itemize}
  \item Волна на тонкой бесконечной пластине --- известные формулы для отражения и пропускания через диэлектрический слой.
  \item Волна, рассеянная бесконечным цилиндром, --- типовой ряд Бесселя--Ханкеля с граничными условиями на границе диэлектрик--вакуум.
\end{itemize}

При необходимости эти решения можно упростить, рассматривая приближения (например, $d\ll\lambda$ для пластины, $k_1 a\ll1$ для тонкой нити), но все вышеприведённые формулы справедливы в общем случае.

\end{document}