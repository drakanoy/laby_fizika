\documentclass{article}
\usepackage[utf8]{inputenc}
\usepackage[russian]{babel}
\usepackage{amsmath, amssymb, graphicx}
\usepackage{geometry}
\usepackage{hyperref}
%\usepackage{physics}
\usepackage{bm}

\begin{document}

\textbf{Решение задачи 4}

\section*{Условие задачи}

Двумерный слой расположен в вакууме в плоскости \( z = 0 \). Вектор поляризуемости слоя задан как
\[
\mathbf{P}(\bm{\rho}, z) = \delta(z)\, \alpha\, \mathbf{E}_{\parallel}(\bm{\rho}),
\]
где \( \bm{\rho} \) — двумерный радиус-вектор в плоскости, \( \alpha \) — скалярная поляризуемость, а \( \mathbf{E}_{\parallel}(\bm{\rho}) \) — электрическое поле в плоскости \( z = 0 \):
\[
\mathbf{E}_{\parallel}(\bm{\rho}) = -\nabla_{\bm{\rho}} \varphi(\bm{\rho}, z = 0).
\]
Найти потенциал, который наводит в слое точечный заряд \( e \), с учётом наведённой поляризуемости слоя.

\section*{Решение}

\subsection*{1. Понимание системы}

Рассматривается точечный заряд \( e \), расположенный в начале координат \( (\bm{\rho} = 0, z = 0) \), рядом с двумерным поляризуемым слоем в плоскости \( z = 0 \). Заряд создаёт электрическое поле, которое наводит в слое поляризацию, влияющую на общий потенциал. Требуется найти полный потенциал \( \varphi(\bm{\rho}, z = 0) \) в плоскости слоя.

\subsection*{2. Формулировка задачи}

Полный потенциал \( \varphi \) на \( z = 0 \) является суммой потенциала от точечного заряда \( \varphi_0 \) и наведённого потенциала \( \varphi_{\text{ind}} \) от поляризации слоя:
\[
\varphi(\bm{\rho}, z = 0) = \varphi_0(\bm{\rho}, z = 0) + \varphi_{\text{ind}}(\bm{\rho}, z = 0).
\]

\subsubsection*{Потенциал от точечного заряда}

В системе СГС потенциал от точечного заряда в двумерном пространстве при \( z = 0 \) равен:
\[
\varphi_0(\bm{\rho}, z = 0) = \frac{e}{\rho},
\]
где \( \rho = |\bm{\rho}| \).

\subsubsection*{Наведённая поверхностная плотность заряда}

Наведённая поверхностная плотность заряда \( \sigma_{\text{ind}} \) связана с поляризацией следующим образом:
\[
\sigma_{\text{ind}}(\bm{\rho}) = -\nabla_{\bm{\rho}} \cdot \mathbf{P}_{\parallel}(\bm{\rho}, z = 0).
\]
Подставляя \( \mathbf{P}_{\parallel}(\bm{\rho}, z = 0) = \alpha \mathbf{E}_{\parallel}(\bm{\rho}) \) и \( \mathbf{E}_{\parallel}(\bm{\rho}) = -\nabla_{\bm{\rho}} \varphi(\bm{\rho}, z = 0) \), получаем:
\[
\sigma_{\text{ind}}(\bm{\rho}) = \alpha \nabla_{\bm{\rho}}^2 \varphi(\bm{\rho}, z = 0).
\]

\subsubsection*{Наведённый потенциал}

Наведённый потенциал на \( z = 0 \) от \( \sigma_{\text{ind}} \) выражается через интеграл:
\[
\varphi_{\text{ind}}(\bm{\rho}, z = 0) = \int \frac{\sigma_{\text{ind}}(\bm{\rho}')}{|\bm{\rho} - \bm{\rho}'|} \, d^2 \rho'.
\]

\subsection*{3. Применение преобразования Фурье}

Для упрощения свёртки воспользуемся двумерным преобразованием Фурье по переменной \( \bm{\rho} \).

\subsubsection*{Определения преобразования Фурье}

Прямое и обратное преобразования Фурье определяются как:
\[
\tilde{f}(\bm{k}) = \int e^{-i \bm{k} \cdot \bm{\rho}} f(\bm{\rho}) \, d^2 \rho,
\]
\[
f(\bm{\rho}) = \frac{1}{(2\pi)^2} \int e^{i \bm{k} \cdot \bm{\rho}} \tilde{f}(\bm{k}) \, d^2 k.
\]

\subsubsection*{Преобразование Фурье потенциала от точечного заряда}

Преобразование Фурье от \( \varphi_0(\bm{\rho}, z = 0) \):
\[
\tilde{\varphi}_0(\bm{k}) = \int e^{-i \bm{k} \cdot \bm{\rho}} \frac{e}{\rho} \, d^2 \rho = 2\pi \frac{e}{k},
\]
где \( k = |\bm{k}| \).

\subsubsection*{Преобразование Фурье наведённого потенциала}

В преобразованном пространстве наведённый потенциал связан с \( \tilde{\sigma}_{\text{ind}}(\bm{k}) \) следующим образом:
\[
\tilde{\varphi}_{\text{ind}}(\bm{k}) = 2\pi \frac{\tilde{\sigma}_{\text{ind}}(\bm{k})}{k}.
\]

\subsubsection*{Преобразование Фурье наведённой плотности заряда}

Используя \( \sigma_{\text{ind}}(\bm{\rho}) = \alpha \nabla_{\bm{\rho}}^2 \varphi(\bm{\rho}, z = 0) \), получаем:
\[
\tilde{\sigma}_{\text{ind}}(\bm{k}) = -\alpha k^2 \tilde{\varphi}(\bm{k}).
\]

\subsubsection*{Общий потенциал в преобразованном пространстве}

Комбинируя результаты:
\[
\tilde{\varphi}(\bm{k}) = \tilde{\varphi}_0(\bm{k}) + \tilde{\varphi}_{\text{ind}}(\bm{k}) = \tilde{\varphi}_0(\bm{k}) - 2\pi \alpha k \tilde{\varphi}(\bm{k}).
\]
Переносим все члены с \( \tilde{\varphi}(\bm{k}) \) в левую часть:
\[
\tilde{\varphi}(\bm{k}) + 2\pi \alpha k \tilde{\varphi}(\bm{k}) = \tilde{\varphi}_0(\bm{k}).
\]
Выносим \( \tilde{\varphi}(\bm{k}) \) за скобки:
\[
\left(1 + 2\pi \alpha k\right) \tilde{\varphi}(\bm{k}) = 2\pi \frac{e}{k}.
\]
Решаем относительно \( \tilde{\varphi}(\bm{k}) \):
\[
\tilde{\varphi}(\bm{k}) = \frac{2\pi e}{k \left(1 + 2\pi \alpha k\right)}.
\]

\subsection*{4. Обратное преобразование Фурье}

Теперь выполним обратное преобразование Фурье для нахождения \( \varphi(\bm{\rho}, z = 0) \).

\subsubsection*{Упрощение с учётом радиальной симметрии}

Благодаря радиальной симметрии, обратное преобразование сводится к интегралу:
\[
\varphi(\rho, z = 0) = \frac{1}{2\pi} \int_0^\infty k \tilde{\varphi}(k) J_0(k \rho) \, dk,
\]
где \( J_0 \) — функция Бесселя первого рода нулевого порядка.

\subsubsection*{Подстановка \( \tilde{\varphi}(k) \)}

Подставляем найденное выражение для \( \tilde{\varphi}(k) \):
\[
\varphi(\rho, z = 0) = \frac{1}{2\pi} \int_0^\infty k \left( \frac{2\pi e}{k \left(1 + 2\pi \alpha k\right)} \right) J_0(k \rho) \, dk.
\]
Упрощаем выражение:
\[
\varphi(\rho, z = 0) = e \int_0^\infty \frac{J_0(k \rho)}{1 + 2\pi \alpha k} \, dk.
\]

\subsubsection*{Введение новой переменной}

Введём замену:
\[
\beta = 2\pi \alpha, \quad s = \beta k \implies k = \frac{s}{\beta}, \quad dk = \frac{ds}{\beta}.
\]
Тогда интеграл принимает вид:
\[
\varphi(\rho, z = 0) = \frac{e}{\beta} \int_0^\infty \frac{J_0\left( \frac{s \rho}{\beta} \right)}{1 + s} \, ds.
\]

\subsection*{5. Применение подсказки}

Согласно подсказке:
\[
\int_0^\infty \frac{J_0(a x)}{x + 1} \, dx = \frac{\pi}{2} \left( H_0(a) - N_0(a) \right),
\]
где \( H_0 \) — функция Струве, а \( N_0 \) — функция Неймана нулевого порядка.

Пусть \( a = \dfrac{\rho}{\beta} \), тогда:
\[
\int_0^\infty \frac{J_0\left( \frac{s \rho}{\beta} \right)}{1 + s} \, ds = \frac{\pi}{2} \left( H_0\left( \frac{\rho}{\beta} \right) - N_0\left( \frac{\rho}{\beta} \right) \right).
\]

\subsection*{6. Итоговое выражение для потенциала}

Подставляем обратно в выражение для \( \varphi(\rho, z = 0) \):
\[
\varphi(\rho, z = 0) = \frac{e}{\beta} \cdot \frac{\pi}{2} \left( H_0\left( \frac{\rho}{\beta} \right) - N_0\left( \frac{\rho}{\beta} \right) \right).
\]
Зная, что \( \beta = 2\pi \alpha \), получаем:
\[
\varphi(\rho, z = 0) = \frac{e}{2\pi \alpha} \cdot \frac{\pi}{2} \left( H_0\left( \frac{\rho}{2\pi \alpha} \right) - N_0\left( \frac{\rho}{2\pi \alpha} \right) \right).
\]
Упрощаем константы:
\[
\varphi(\rho, z = 0) = \frac{e}{4 \alpha} \left( H_0\left( \frac{\rho}{2\pi \alpha} \right) - N_0\left( \frac{\rho}{2\pi \alpha} \right) \right).
\]

\section*{Ответ}

Явное выражение для потенциала:
\[
\boxed{\varphi(\rho, z = 0) = \dfrac{e}{4\alpha} \left[ H_0\left( \dfrac{\rho}{2\pi \alpha} \right) - N_0\left( \dfrac{\rho}{2\pi \alpha} \right) \right]}
\]
где:
\begin{itemize}
    \item \( H_0 \) — \textbf{функция Струве} нулевого порядка.
    \item \( N_0 \) — \textbf{функция Неймана} (функция Бесселя второго рода) нулевого порядка.
    \item \( \alpha \) — константа поляризуемости.
\end{itemize}

\end{document}