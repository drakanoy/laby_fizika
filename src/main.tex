\documentclass{article}
\usepackage[utf8]{inputenc}
\usepackage[russian]{babel}
\usepackage{amsmath}
\usepackage{amssymb}

\begin{document}

\textbf{Задача. Бусина на кольце}

Маленькая бусина массой \( m \) свободно скользит по проволочному кольцу радиуса \( R \). Кольцо вращают с постоянной угловой скоростью \( \omega \) вокруг оси, проходящей через диаметр и параллельной силе тяжести. Найти положения равновесия бусины, исследовать их устойчивость и найти частоту малых колебаний.

\textbf{Решение:}

\textbf{1. Условие задачи:}

- Кольцо расположено в вертикальной плоскости и вращается вокруг вертикальной оси, проходящей через его центр.
- Бусина может свободно двигаться без трения по кольцу.
- Требуется найти положения равновесия, исследовать их устойчивость и определить частоту малых колебаний.

\textbf{2. Выбор системы координат:}

Обозначим угол \(\theta\) как угол отклонения бусины от нижней точки кольца, измеренный против часовой стрелки. Тогда положение бусины задаётся координатами:

\[
x = R \sin \theta, \quad z = -R \cos \theta.
\]

\textbf{3. Силы, действующие на бусину:}

1. \textbf{Сила тяжести} \( \mathbf{F}_g \):

\[
\mathbf{F}_g = -mg \hat{z}.
\]

2. \textbf{Центробежная сила} \( \mathbf{F}_c \) в системе отсчёта, вращающейся вместе с кольцом:

\[
\mathbf{F}_c = m \omega^2 R \sin \theta \, \hat{x}.
\]

\textbf{4. Эффективная потенциальная энергия:}

Эффективная потенциальная энергия в вращающейся системе отсчёта:

\[
U(\theta) = U_g + U_c = -mgR \cos \theta - \dfrac{1}{2} m \omega^2 R^2 \sin^2 \theta.
\]

\textbf{5. Поиск положений равновесия:}

Равновесие достигается там, где первая производная потенциальной энергии по \(\theta\) равна нулю:

\[
\dfrac{dU}{d\theta} = 0.
\]

Вычисляем производную:

\[
\dfrac{dU}{d\theta} = mgR \sin \theta - m \omega^2 R^2 \sin \theta \cos \theta = mR \sin \theta \left( g - \omega^2 R \cos \theta \right).
\]

Приравниваем к нулю:

\[
\sin \theta \left( g - \omega^2 R \cos \theta \right) = 0.
\]

Получаем два случая:

1. \(\sin \theta = 0\):

\[
\theta = 0, \quad \theta = \pi.
\]

2. \(g - \omega^2 R \cos \theta = 0\):

\[
\cos \theta = \dfrac{g}{\omega^2 R}.
\]

Это решение имеет физический смысл, если \(|\cos \theta| \leq 1\), то есть при \(\omega^2 R \geq g\).

\textbf{6. Исследование устойчивости:}

Устойчивость определяется знаком второй производной потенциальной энергии:

\[
\dfrac{d^2 U}{d\theta^2} = mgR \cos \theta - m \omega^2 R^2 (\cos^2 \theta - \sin^2 \theta).
\]

\textbf{a) При \(\theta = 0\):}

\[
\dfrac{d^2 U}{d\theta^2} \bigg|_{\theta = 0} = mgR - m \omega^2 R^2 = mR (g - \omega^2 R).
\]

- Если \(\omega^2 R < g\), то \(\dfrac{d^2 U}{d\theta^2} > 0\) — положение устойчиво.
- Если \(\omega^2 R > g\), то \(\dfrac{d^2 U}{d\theta^2} < 0\) — положение неустойчиво.

\textbf{b) При \(\theta = \pi\):}

\[
\dfrac{d^2 U}{d\theta^2} \bigg|_{\theta = \pi} = -mgR - m \omega^2 R^2 = -mR (g + \omega^2 R) < 0.
\]

- Положение всегда неустойчиво.

\textbf{c) При \(\cos \theta = \dfrac{g}{\omega^2 R}\):}

Вычисляем вторую производную:

\[
\begin{aligned}
\dfrac{d^2 U}{d\theta^2} &= mgR \cos \theta - m \omega^2 R^2 (\cos^2 \theta - \sin^2 \theta) \\
&= mgR \cos \theta - m \omega^2 R^2 (2 \cos^2 \theta - 1).
\end{aligned}
\]

Подставляем \(\cos \theta = \dfrac{g}{\omega^2 R}\):

\[
\cos^2 \theta = \left( \dfrac{g}{\omega^2 R} \right)^2, \quad \sin^2 \theta = 1 - \cos^2 \theta = 1 - \left( \dfrac{g}{\omega^2 R} \right)^2.
\]

Вычисляем:

\[
\dfrac{d^2 U}{d\theta^2} = mgR \left( \dfrac{g}{\omega^2 R} \right ) - m \omega^2 R^2 \left( 2 \left( \dfrac{g}{\omega^2 R} \right )^2 - 1 \right ).
\]

Упрощаем:

\[
\dfrac{d^2 U}{d\theta^2} = m \left( \dfrac{g^2}{\omega^2} \right ) - m \left( \dfrac{2 g^2}{\omega^2} - \omega^2 R^2 \right ) = m \left( \omega^2 R^2 - \dfrac{g^2}{\omega^2} \right ) > 0.
\]

Поскольку \(\omega^2 R^2 - \dfrac{g^2}{\omega^4 R^2} > 0\) при \(\omega^2 R > g\).

- Положение устойчиво при \(\omega^2 R > g\).

\textbf{7. Нахождение частоты малых колебаний:}

\textbf{a) Вокруг \(\theta = 0\):}

Частота малых колебаний определяется выражением:

\[
\omega_0 = \sqrt{ \dfrac{1}{m R^2} \dfrac{d^2 U}{d\theta^2} } = \sqrt{ \dfrac{g - \omega^2 R}{R} }.
\]

- Частота определена только при \(\omega^2 R < g\).

\textbf{b) Вокруг \(\cos \theta_0 = \dfrac{g}{\omega^2 R}\):}

Рассмотрим малые отклонения \(\varphi\) от положения равновесия:

\[
\theta = \theta_0 + \varphi, \quad \varphi \ll 1.
\]

Уравнение движения:

\[
m R^2 \ddot{\varphi} = -\dfrac{d^2 U}{d\theta^2} \bigg|_{\theta = \theta_0} \varphi.
\]

Ранее мы получили:

\[
\dfrac{d^2 U}{d\theta^2} \bigg|_{\theta = \theta_0} = m \left( \omega^2 R^2 - \dfrac{g^2}{\omega^2} \right ).
\]

Уравнение принимает вид:

\[
m R^2 \ddot{\varphi} + m \left( \omega^2 R^2 - \dfrac{g^2}{\omega^2} \right ) \varphi = 0.
\]

Сокращаем на \(m\):

\[
R^2 \ddot{\varphi} + \left( \omega^2 R^2 - \dfrac{g^2}{\omega^2} \right ) \varphi = 0.
\]

Делим на \(R^2\):

\[
\ddot{\varphi} + \left( \omega^2 - \dfrac{g^2}{\omega^4 R^2} \right ) \varphi = 0.
\]

Получаем выражение для частоты малых колебаний:

\[
\omega_0^2 = \omega^2 - \left( \dfrac{g}{\omega R} \right )^2.
\]

Поскольку \(\cos \theta_0 = \dfrac{g}{\omega^2 R}\), то \(\sin^2 \theta_0 = 1 - \left( \dfrac{g}{\omega^2 R} \right )^2\). Тогда:

\[
\omega_0 = \omega \sqrt{ 1 - \left( \dfrac{g}{\omega^2 R} \right )^2 } = \omega \sin \theta_0.
\]

\textbf{8. Итоговые результаты:}

- Положения равновесия:

  - \(\theta = 0\): устойчиво при \(\omega^2 R < g\).
  - \(\theta = \pi\): всегда неустойчиво.
  - \(\cos \theta = \dfrac{g}{\omega^2 R}\): устойчиво при \(\omega^2 R > g\).

- Частоты малых колебаний:

  - Вокруг \(\theta = 0\):

    \[
    \omega_0 = \sqrt{ \dfrac{g - \omega^2 R}{R} }.
    \]

  - Вокруг \(\cos \theta = \dfrac{g}{\omega^2 R}\):

    \[
    \omega_0 = \omega \sqrt{ 1 - \left( \dfrac{g}{\omega^2 R} \right )^2 } = \omega \sin \theta_0.
    \]

\textbf{Заключение:}

Бусина имеет устойчивые положения равновесия в нижней точке кольца при низкой скорости вращения (\(\omega^2 R < g\)) и при углах, удовлетворяющих условию \(\cos \theta = \dfrac{g}{\omega^2 R}\), при высокой скорости вращения (\(\omega^2 R > g\)). Частота малых колебаний зависит от угловой скорости вращения кольца, радиуса кольца и ускорения свободного падения.

\end{document}