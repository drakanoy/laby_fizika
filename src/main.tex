\documentclass{article}
\usepackage[utf8]{inputenc}
\usepackage[russian]{babel}
\usepackage{amsmath}
\usepackage{amssymb}
\usepackage{textcomp}

\begin{document}

\section*{Задача 2}
Пусть $G$ --- связная группа Ли. Пусть
\[
p \colon \widetilde{G} \to G
\]
её универсальное накрытие. Доказать, что существует структура группы Ли на $\widetilde{G}$, т.~ч.
\begin{enumerate}
\item $p \colon \widetilde{G} \to G$ --- гладкая функция,
\item $p \colon \widetilde{G} \to G$ является гомоморфизмом групп Ли.
\end{enumerate}

\noindent
\textbf{Доказательство.} Обозначим умножение на $G$ как
\[
\mu \colon G \times G \to G,
\]
взятие обратного как
\[
\mathrm{inv} \colon G \to G.
\]
Порядок доказательства:
\begin{enumerate}
\item Введём на $\widetilde{G}$ структуру гладкого многообразия.
\item Покажем, что $p \colon \widetilde{G} \to G$ гладкая.
\item Введём на $\widetilde{G}$ операции умножения
\[
\widetilde{\mu}(x,y) = xy
\]
и взятия обратного
\[
\widetilde{\mathrm{inv}}(x) = x^{-1}.
\]
\item Проверим, что $\widetilde{\mu}$ и $\widetilde{\mathrm{inv}}$ гладкие, откуда следует, что на $\widetilde{G}$ получается структура группы Ли.
\item Докажем, что $p \colon \widetilde{G} \to G$ является гомоморфизмом групп Ли.
\end{enumerate}

\subsection*{1) Топологические свойства $\widetilde{G}$}
Уже знаем (из построения универсального накрытия на лекциях), что $\widetilde{G}$ --- топологическое пространство. Нужно проверить выполнение трёх свойств из определения гладкого многообразия:

\paragraph{1.1) $\widetilde{G}$ --- хаусдорфово.}
Рассмотрим две точки $x, y \in \widetilde{G}$. Пусть $p(x) = a$ и $p(y) = b$ в $G$.

\begin{itemize}
\item Если $p(x) = p(y) = a$, то существует <<хорошая>> окрестность $U(a)$ так, что $p \lvert_{V_x}$ --- гомеоморфизм $x$, а $x$ и $y$ лежат на разных $V_i, V_j$, $i \neq j$. Это и будут непересекающиеся окрестности.
\item Если $p(x) = a \neq b = p(y)$, то существуют непересекающиеся хорошие окрестности $U(a)$ и $U(b)$: $U(a) \cap U(b) = \varnothing$. Тогда можно выбрать окрестности
\[
x \in V(x) \subset p^{-1}(U(a)), \quad
y \in W(y) \subset p^{-1}(U(b)),
\]
которые также будут непересекающимися. В силу того, что $p \lvert_{V(x)}$ и $p \lvert_{W(y)}$ --- гомеоморфизмы, мы получаем, что $V(x) \cap W(y) = \varnothing$.
\end{itemize}

Поскольку $G$ --- хаусдорфово, мы можем добиваться непересекаемости окрестностей образов, и при прообразах через $p$ также получаем непересекающиеся окрестности на $\widetilde{G}$.

\paragraph{1.2) $\widetilde{G}$ имеет счётную базу.}
Покроем $G$ хорошими окрестностями. Из них можно выбрать счётный набор $\{U_i\}$, так как в $G$ существует счётная база. На лекциях показано, что
\[
p^{-1}(U_i) = \bigsqcup_j U_{i,j}
\]
--- счётное дизъюнктное объединение $U_{i,j} \subset \widetilde{G}$. Таким образом, $\{U_{i,j}\}$ --- счётная база в $\widetilde{G}$.

\paragraph{1.3) Построение атласа.}
Осталось предъявить атлас. Рассмотрим точку $x \in \widetilde{G}$, $p(x) = a$. Знаем, что существует хорошая окрестность $U(a)$ в $G$, такая, что $p \lvert_{V_x}$ --- гомеоморфизм для $V_x = p^{-1}(U(a)) \cap$ (некоторая компонента).

Так как $G$ имеет структуру гладкого многообразия размерности $2n$, можно покрыть $U = U_i \cap U_j$ набором карт $\{\varphi_i: U_i \to W_i \subset \mathbb{R}^n\}$. Этому набору соответствует набор
\[
V_i = p^{-1}(U_i \cap U),
\]
покрывающий $V$. Тогда определим атлас на $\widetilde{G}$ так:
\[
\widetilde{\varphi}_i = \varphi_i \circ p \lvert_{V_i} \colon
p^{-1}(U_i \cap U) \to W_i \subset \mathbb{R}^n.
\]
Отображение $\widetilde{\varphi}_i$ является композицией гомеоморфизмов, следовательно --- это допустимая карта. Чтобы задать атлас $\mathcal{A}_{\widetilde{G}}$, перебираем все $x$ в $\widetilde{G}$, берём соответствующие хорошие окрестности (их счётное число) и карты из атласа $\mathcal{A}_G$. Так, очевидно, получаем, что
\[
\widetilde{G} = \bigcup_i V_i.
\]
Согласованность $\widetilde{\varphi}_i$ вытекает из согласованности $\varphi_i$ на $G$.

Итак, $\widetilde{G}$ имеет структуру гладкого многообразия.

\subsection*{2) Гладкость отображения $p \colon \widetilde{G} \to G$}
Покажем, что $p \colon \widetilde{G} \to G$ --- гладкая. Это значит, что каждый локальный представитель $p$ в картах является гладким. Пусть
\[
\varphi_j \circ p \circ (\widetilde{\varphi}_i)^{-1} \colon W_i \to W_j
\]
--- это композиция гладких отображений (так как $\widetilde{\varphi}_i$, $\varphi_j$ согласованные карты), откуда $p$ гладкое.

\subsection*{3) Введение операций $\widetilde{\mu}$ и $\widetilde{\mathrm{inv}}$}
Определим
\[
\widetilde{\mu} \colon \widetilde{G} \times \widetilde{G} \to \widetilde{G},
\quad
\widetilde{\mu}\bigl([\alpha(t)], [\beta(t)]\bigr) = [\alpha(t) \cdot \beta(t)],
\]
где $[\alpha(t)]$ --- класс гомотопной петли в $G$. Проверим корректность: если
\[
\alpha \sim \alpha', \quad \beta \sim \beta',
\]
то
\[
\alpha(t)\cdot \beta(t) \sim \alpha'(t)\cdot \beta'(t).
\]
Отсюда корректность операции $\widetilde{\mu}$.

Аналогично вводим
\[
\widetilde{\mathrm{inv}} \colon \widetilde{G} \to \widetilde{G}, \quad
\widetilde{\mathrm{inv}}\bigl([\alpha(t)]\bigr) = [\alpha(t)]^{-1} = \bigl[\alpha(t)^{-1}\bigr].
\]
Здесь $[\alpha(t)]^{-1}$ --- путь в $G$, равный в каждой точке обратному элементу $\alpha(t)$.

Так на $\widetilde{G}$ получаем структуру группы: нейтральным элементом служит
\[
[e_c(t)],
\]
где $e_c(t)$ --- тривиальный (константный) путь в нейтральном элементе $G$. Умножение есть
\[
[\alpha(t)] \cdot [e_c(t)] = [\alpha(t)\cdot e_c(t)] = [\alpha(t)].
\]

\subsection*{4) Гладкость $\widetilde{\mu}$ и $\widetilde{\mathrm{inv}}$}
Чтобы $\widetilde{G}$ стало группой Ли, нужно показать, что умножение $\widetilde{\mu}$ и обращение $\widetilde{\mathrm{inv}}$ являются гладкими отображениями.

\paragraph{4.1) Гладкость $\widetilde{\mu}$.}
Пусть $\mu \colon G \times G \to G$ гладкое многообразие. Аналогичные карты задаются и для $\widetilde{G} \times \widetilde{G}$. С учётом того, что
\[
\mu = p \circ \widetilde{\mu} \circ (p^{-1}\times p^{-1}),
\]
мы можем проверить локально: в картах $\widetilde{\varphi}_i$ и $\widetilde{\varphi}_j$ композиция с $\mu$ остаётся гладкой, так как само $\mu$ гладко на $G \times G$.

\paragraph{4.2) Гладкость $\widetilde{\mathrm{inv}}$.}
Аналогично,
\[
\mathrm{inv} = p \circ \widetilde{\mathrm{inv}} \circ p^{-1},
\]
а $\mathrm{inv}$ гладко на $G$. Локально это сводится к композиции диффеоморфизмов в картах, значит, $\widetilde{\mathrm{inv}}$ гладко.

Таким образом, на $\widetilde{G}$ построена структура группы Ли.

\subsection*{5) Гомоморфизм групп Ли}
Осталось показать, что $p \colon \widetilde{G} \to G$ является гомоморфизмом групп Ли. Мы уже знаем, что $p$ гладкое (пункт 2). Проверим гомоморфизм:
\[
p\bigl([\alpha(t)] \cdot [\beta(t)]\bigr) = p\bigl([\alpha(t)\,\beta(t)]\bigr)
= (\alpha(1)\cdot \beta(1)) = p\bigl([\alpha(t)]\bigr) \cdot p\bigl([\beta(t)]\bigr).
\]
Таким образом, $p$ --- гомоморфизм групп Ли.



\section*{Задача 2 (альтернативная формулировка доказательства)}

Пусть $G$ --- связная группа Ли, а
\[
p \colon \widetilde{G} \to G
\]
есть её универсальное накрытие. Требуется показать, что на $\widetilde{G}$ можно ввести структуру группы Ли, причём:
\begin{enumerate}
\item отображение $p \colon \widetilde{G} \to G$ является гладким;
\item то же самое $p$ даёт гомоморфизм групп Ли.
\end{enumerate}

\noindent
\textbf{План решения.} Пусть умножение в $G$ задаётся отображением
\[
\mu \colon G \times G \to G
\]
и обращение ---
\[
\mathrm{inv} \colon G \to G.
\]
Тогда докажем следующее:

\begin{enumerate}
\item Снабдим $\widetilde{G}$ структурой гладкого многообразия.
\item Убедимся, что $p \colon \widetilde{G} \to G$ является гладкой функцией.
\item Определим на $\widetilde{G}$ операции умножения
\[
\widetilde{\mu}(x,y) = x y
\]
и взятия обратного
\[
\widetilde{\mathrm{inv}}(x) = x^{-1}.
\]
\item Покажем, что данные операции гладкие, откуда $\widetilde{G}$ становится группой Ли.
\item Наконец, проверим, что $p \colon \widetilde{G} \to G$ есть гомоморфизм групп Ли.
\end{enumerate}

\subsection*{1) Топология на $\widetilde{G}$}
Из построения универсального накрытия известно, что $\widetilde{G}$ --- топологическое пространство. Нужно лишь подтвердить три условия для гладкого многообразия:

\paragraph{1.1) Хаусдорфовость.}
Возьмём произвольные точки $x,y \in \widetilde{G}$. Пусть $p(x)=a$ и $p(y)=b$ в $G$.
\begin{itemize}
\item Если $a=b$, то берём в $G$ <<хорошую>> окрестность $U(a)$, на которой $p$ является гомеоморфизмом на соответствующих компонентах прообраза. Точки $x$ и $y$ попадают в разные компоненты, которые можно выбрать как непересекающиеся.
\item Если $a \neq b$, то в $G$ найдутся непересекающиеся <<хорошие>> окрестности $U(a)$ и $U(b)$, а в $\widetilde{G}$ --- их прообразы, которые тоже непересекаются благодаря тому, что $p$ локально является гомеоморфизмом.
\end{itemize}
Поскольку $G$ --- хаусдорфово, то и в $\widetilde{G}$ строятся непересекающиеся окрестности соответствующих точек.

\paragraph{1.2) Счётная база.}
Выберем в $G$ счётный набор <<хороших>> окрестностей $\{U_i\}$, который возможен благодаря тому, что $G$ имеет счётную базу. Тогда прообраз каждой $U_i$ под $p$ раскладывается в дизъюнктное объединение $U_{i,j}$:
\[
p^{-1}(U_i) = \bigsqcup_j U_{i,j}.
\]
Все такие $U_{i,j}$ вместе образуют счётную базу на $\widetilde{G}$.

\paragraph{1.3) Атлас на $\widetilde{G}$.}
Пусть $x \in \widetilde{G}$ и $p(x)=a$. Так как $a$ имеет <<хорошую>> окрестность $U(a)$ в $G$ и $p$ при этом локальный гомеоморфизм, рассмотрим соответствующий кусок $V_x \subset \widetilde{G}$, который отображается гомеоморфно на $U(a)$.

В $G$ уже есть гладкая структура с атласом $\{\varphi_i\}$. Тогда на каждом $V_x$ зададим карты
\[
\widetilde{\varphi}_i = \varphi_i \circ p \bigl|_{V_x}.
\]
Композиция гомеоморфизма и гладкой карты даёт гладкую карту. Собирая такие карты по всем <<хорошим>> окрестностям и их прообразам, получаем полный атлас на $\widetilde{G}$. Согласованность новых карт получается из согласованности исходных $\varphi_i$.

\subsection*{2) Гладкость $p \colon \widetilde{G} \to G$}
Чтобы проверить гладкость $p$, достаточно проверить гладкость его локальных версий в координатных картах. Но локально $p$ --- композиция уже гладких (или гомеоморфных) отображений, следовательно, гладкое.

\subsection*{3) Групповая операция и обращение}
На $\widetilde{G}$ определяется умножение
\[
\widetilde{\mu}\bigl([\alpha(t)], [\beta(t)]\bigr) = \bigl[\alpha(t)\,\beta(t)\bigr],
\]
где $[\alpha(t)]$ и $[\beta(t)]$ --- классы петель в $G$. Корректность: гомотопные петли при умножении дают гомотопные результаты. Нейтральным элементом выступает класс $\bigl[e_c(t)\bigr]$, где $e_c(t)$ --- постоянный путь в нейтральном элементе $G$. Обращение петли $[\alpha(t)]$ задаётся путём
\[
[\alpha(t)]^{-1} = \bigl[\alpha(t)^{-1}\bigr].
\]

\subsection*{4) Гладкость $\widetilde{\mu}$ и $\widetilde{\mathrm{inv}}$}
\paragraph{4.1) Гладкость $\widetilde{\mu}$.}
Умножение на $G$ есть гладкое отображение $\mu \colon G \times G \to G$. На $\widetilde{G} \times \widetilde{G}$ по определению
\[
\mu \;=\; p \;\circ\; \widetilde{\mu}\;\circ\;(p^{-1} \times p^{-1}),
\]
и поскольку локально все карты согласованы с картами в $G \times G$, операция $\widetilde{\mu}$ получается гладкой.

\paragraph{4.2) Гладкость $\widetilde{\mathrm{inv}}$.}
Аналогично, $\mathrm{inv} \colon G \to G$ гладко. Тогда
\[
\mathrm{inv} \;=\; p \;\circ\; \widetilde{\mathrm{inv}} \;\circ\; p^{-1}.
\]
В локальных координатах эта композиция диффеоморфизмов также гладкая, значит, и $\widetilde{\mathrm{inv}}$ гладко.

Таким образом, на $\widetilde{G}$ формируется структура группы Ли.

\subsection*{5) Гомоморфизм групп Ли}
Наконец, покажем, что $p \colon \widetilde{G} \to G$ --- гомоморфизм групп Ли. Мы уже доказали его гладкость. Для свойств гомоморфизма надо лишь проверить согласованность умножений:
\[
p\Bigl([\alpha(t)] \cdot [\beta(t)]\Bigr) \;=\; p\Bigl([\alpha(t)\,\beta(t)]\Bigr)
\;=\; \alpha(1)\,\beta(1)
\;=\; p([\alpha(t)])\;\cdot\;p([\beta(t)]).
\]
Значит, $p$ действительно переводит произведение в произведение, то есть является гомоморфизмом групп Ли.

\end{document}