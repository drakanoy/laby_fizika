\documentclass{article}
\usepackage[utf8]{inputenc}
\usepackage[russian]{babel}
\usepackage{amsmath}
\usepackage{amssymb}
\usepackage{textcomp}


\begin{document}

\section*{Решение интеграла и свойства гамма-функции}

Рассмотрим следующий интеграл:
\[
I = \int_0^{+\infty} dx \int_x^{+\infty} dy\, e^{-y} y^{2025} x^{2007} \left(1 - \frac{x}{y} \right)^{19}
\]

\subsection*{1. Смена порядка интегрирования}

Область интегрирования:
\[
x \in [0, \infty), \quad y \in [x, \infty)
\]
Поменяем порядок интегрирования:
\[
y \in [0, \infty), \quad x \in [0, y]
\]
Тогда:
\[
I = \int_0^{+\infty} dy\, e^{-y} y^{2025} \int_0^y dx\, x^{2007} \left(1 - \frac{x}{y} \right)^{19}
\]

\subsection*{2. Замена переменной}

Пусть \( u = \frac{x}{y} \Rightarrow x = uy, \, dx = y\, du \)

\begin{align*}
\int_0^y x^{2007} \left(1 - \frac{x}{y} \right)^{19} dx
&= y^{2008} \int_0^1 u^{2007} (1 - u)^{19} du \\
&= y^{2008} \cdot B(2008, 20)
\end{align*}

Где бета-функция:
\[
B(a, b) = \frac{\Gamma(a)\Gamma(b)}{\Gamma(a + b)}
\]

\[
B(2008, 20) = \frac{\Gamma(2008)\Gamma(20)}{\Gamma(2028)}
\]

\subsection*{3. Финальная подстановка}

\begin{align*}
I &= \int_0^{+\infty} dy\, e^{-y} y^{2025} \cdot y^{2008} \cdot \frac{\Gamma(2008)\Gamma(20)}{\Gamma(2028)} \\
&= \frac{\Gamma(2008)\Gamma(20)}{\Gamma(2028)} \int_0^{+\infty} y^{4033} e^{-y} dy \\
&= \frac{\Gamma(2008)\Gamma(20)}{\Gamma(2028)} \cdot \Gamma(4034)
\end{align*}

\[
\boxed{I = \Gamma(4034) \cdot \frac{\Gamma(2008)\Gamma(20)}{\Gamma(2028)}}
\]

\subsection*{4. Повтор с новой степенью при \texorpdfstring{$x^{2008}$}{x\^{}2008}}

Аналогично:

\[
\boxed{I = \Gamma(4035) \cdot \frac{\Gamma(2009)\Gamma(20)}{\Gamma(2029)}}
\]

\subsection*{5. Свойство гамма-функции}

\[
\Gamma(n + 1) = n \cdot \Gamma(n) \Rightarrow \frac{\Gamma(n + 1)}{\Gamma(n)} = \boxed{n}
\]

\section*{Решение задачи про чувака в очереди за кофэ}
решил сам через сумму прогрессии на планшете (ну там еще пара значений из чата где неправильное решение. посмотрим)
вот что пишет deepseak:

\section*{Вероятность неполучения кофе за 19 минут}

\textbf{Условие задачи:} Андрей стоит 18-ым в очереди из 18 человек. Каждую минуту:
\begin{itemize}
    \item С вероятностью \(0.05k\) (где \(k\) — текущее количество людей перед ним) в очередь добавляется новый человек, и один уходит (очередь остаётся длины \(k\)).
    \item С вероятностью \(1 - 0.05k\) очередь уменьшается на 1 человека (становится \(k-1\)).
\end{itemize}
Требуется найти вероятность, что через 19 минут перед Андреем останется хотя бы 1 человек.

\subsection*{Решение методом динамического программирования}

\textbf{Шаг 1. Определение состояний и начальных условий} \\
Обозначим:
\[
P(t, k) = \text{вероятность, что через \(t\) минут перед Андреем \(k\) человек}.
\]
Начальное условие:
\[
P(0, 18) = 1, \quad P(0, k) = 0 \quad \text{для всех } k \neq 18.
\]

\textbf{Шаг 2. Рекуррентные соотношения} \\
Для \(t \geq 1\) и \(k \geq 1\):
\[
P(t, k) = \underbrace{0.05k \cdot P(t-1, k)}_{\text{осталось }k} + \underbrace{(1 - 0.05(k+1)) \cdot P(t-1, k+1)}_{\text{перешло из }k+1}.
\]
Для \(k = 0\):
\[
P(t, 0) = 0 \quad (\text{Андрей уже получил кофе}).
\]

\textbf{Шаг 3. Вычисление вероятностей} \\
Итеративно вычисляем \(P(t, k)\) для:
\[
t = 1, 2, \ldots, 19 \quad \text{и} \quad k = 0, 1, \ldots, 18.
\]

\textbf{Шаг 4. Итоговая вероятность} \\
После 19 шагов вероятность, что очередь не опустела:
\[
P_{\text{неуспел}} = \sum_{k=1}^{18} P(19, k).
\]

\subsection*{Численный результат}
Используя точные вычисления с округлением до \(10^{-9}\), получаем:
\[
P_{\text{неуспел}} = \boxed{0.3792705799}.
\]


\section*{Условие задачи про дисперсию пути на марсе}

На поверхность Марса, представленную идеальным шаром единичного радиуса,
высаживается марсоход.
Из--за ошибки ПО он приземляется не около станции (пусть она расположена в северном полюсе),
а в случайной точке $P$, равномерно распределённой по поверхности.
Ровер движется к станции по кратчайшему пути со \textbf{скоростью} $v=1$~у.е./ч.
Найти \emph{дисперсию времени} $T$, которое потребуется марсоходу,
и дать численное значение с точностью не хуже $10^{-9}$.

\section{Математическая модель}

\begin{itemize}
  \item Шар радиуса $R=1$ центра $C$.
  \item Точка $O$ --- северный полюс (координаты $(0,0,1)$).
  \item Точка посадки $P$ равномерна на $\mathbb S^{2}$.
  \item Центральный угол между $OP$ равен $\theta\in[0,\pi]$.
  \item Кратчайшая геодезическая длины $s=R\theta=\theta$.
  \item При $v=1$ время пути совпадает с углом: $T=\theta$.
\end{itemize}

\section{Распределение угла \texorpdfstring{$\theta$}{θ}}

Площадь элементарного пояса ширины $d\theta$ на сфере
\[
  dS = 2\pi R^{2}\sin\theta\,d\theta = 2\pi\sin\theta\,d\theta .
\]
Так как плотность равномерна, вероятность попасть в этот пояс
\[
  dP = \frac{dS}{4\pi R^{2}}=\frac12\sin\theta\,d\theta .
\]
Отсюда плотность
\[
  f_{\theta}(\theta)=\frac12\sin\theta,\qquad 0\le\theta\le\pi.
\]

\section{Моменты распределения}

\subsection{Математическое ожидание}

\[
  \mathbb E[\theta]=\int_{0}^{\pi} \theta\,\frac12\sin\theta\,d\theta
  =\frac12\Bigl[-\theta\cos\theta+\sin\theta\Bigr]_{0}^{\pi}
  =\frac{\pi}{2}.
\]

\subsection{Второй момент}

Интегрируя по частям дважды:
\[
  \mathbb E[\theta^{2}]
  =\int_{0}^{\pi}\theta^{2}\,\frac12\sin\theta\,d\theta
  =\frac12\left(\pi^{2}-4\right)
  =\frac{\pi^{2}}{2}-2.
\]

\section{Дисперсия времени пути}

\[
  \operatorname{Var}(T)=\mathbb E[\theta^{2}]-
        \bigl(\mathbb E[\theta]\bigr)^{2}
        =\left(\frac{\pi^{2}}{2}-2\right)-\left(\frac{\pi}{2}\right)^{2}
        =\frac{\pi^{2}}{4}-2.
\]

\section{Численное значение}

\[
  \frac{\pi^{2}}{4}-2 \approx 0.4674011002723396547.
\]

Разница между любым значением в $10^{-9}$ окрестности и истинным результатом
меньше требуемого порога.

\section*{Проверка 1--D аналога (круг)}

Для окружности $\mathbb S^{1}$ (1--мерный аналог) при равномерном выборе точки
угол $\phi$ распределён равномерно на $(0,2\pi)$.
Тогда $\mathbb E[\phi]=\pi$ и $\operatorname{Var}(\phi)=\pi^{2}/3$.
Сравнение подчёркивает, что фактор $\sin\theta$ в 3D приводит к другой дисперсии.



\section*{Решение задачи нахождения радиуса сходимости ряда}

Дано: ряд \(\sum_{n=1}^\infty \frac{x^{\lfloor \sqrt{2n \ln \ln n} \cdot \ln n \rfloor}}{(\lfloor \sqrt{2n \ln \ln n} \rfloor)!}\).

\textbf{Шаг 1. Асимптотика показателя степени.}\\
Для больших \(n\):
\[
k_n = \lfloor \sqrt{2n \ln \ln n} \cdot \ln n \rfloor \approx \sqrt{2n \ln \ln n} \cdot \ln n.
\]

\textbf{Шаг 2. Асимптотика знаменателя.}\\
\[
m_n = \lfloor \sqrt{2n \ln \ln n} \rfloor \approx \sqrt{2n \ln \ln n}.
\]

\textbf{Шаг 3. Оценка факториала по формуле Стирлинга.}\\
\[
m_n! \approx \sqrt{2\pi m_n} \left(\frac{m_n}{e}\right)^{m_n}.
\]
Тогда коэффициент ряда:
\[
a_n = \frac{1}{m_n!} \approx \frac{1}{\sqrt{2\pi m_n}} \left(\frac{e}{m_n}\right)^{m_n}.
\]

\textbf{Шаг 4. Вычисление \(\limsup |a_n|^{1/k_n}\).}\\
\[
|a_n|^{1/k_n} \approx \left( \frac{1}{\sqrt{2\pi m_n}} \left(\frac{e}{m_n}\right)^{m_n} \right)^{1/k_n}.
\]
Пренебрегая медленно растущими членами:
\[
\left(\frac{e}{m_n}\right)^{m_n/k_n}.
\]
Подставляя \(m_n/k_n \approx \frac{\sqrt{2n \ln \ln n}}{\sqrt{2n \ln \ln n} \cdot \ln n} = \frac{1}{\ln n}\):
\[
\left(\frac{e}{\sqrt{2n \ln \ln n}}\right)^{1/\ln n}.
\]
Логарифмируя и упрощая:
\[
\exp\left( \frac{1 - \frac{1}{2} \ln(2n \ln \ln n)}{\ln n} \right) \approx e^{-1/2}.
\]

\textbf{Шаг 5. Радиус сходимости.}\\
\[
\frac{1}{R} = \limsup |a_n|^{1/k_n} = e^{-1/2} \implies R = e^{1/2} = \sqrt{e}.
\]

\boxed{\sqrt{e}}


%%%%%%%%%%%%%%%%%%%%%%%%%%%%%%%%%%%%%%%%%%%%%%%%%%%%%%%%%%%%%%%%%%%%%%%%%%%%%%
\section{Радиус сходимости ряда}

\subsection\*{Формулировка}
Найти радиус сходимости степенного ряда

$$
  \sum_{n=1}^{\infty} \frac{x^{\bigl\lfloor\sqrt{2n\ln\ln n\,\ln n}\bigr\rfloor}}{\bigl(\lfloor\sqrt{2n\ln\ln n}\rfloor\bigr)!}.
$$

\subsection\*{Решение}
\paragraph{1.  Обозначения.} Положим

$$
  p_n:=\Bigl\lfloor\sqrt{2n\ln\ln n}\Bigr\rfloor,\qquad
  q_n:=\Bigl\lfloor\sqrt{2n\ln\ln n\,\ln n}\Bigr\rfloor.
$$

Тогда коэффициенты ряда равны $c\_{q\_n}=1/p\_n!$, а все остальные $c\_k=0$.

\paragraph{2.  Асимптотика.} Отбрасывая целые части (они не влияют на порядок), получаем $q\_n\sim p\_n\sqrt{\ln n}$ при $n\to\infty$.
По формуле Стирлинга

$$
  \ln(p_n!) = p_n\ln p_n - p_n + \mathcal O(\ln p_n).
$$

Следовательно

$$
  \frac{\ln|c_{q_n}|}{q_n}= -\frac{p_n\ln p_n - p_n + \mathcal O(\ln p_n)}{p_n\sqrt{\ln n}}
  = -\frac{\ln p_n - 1 + o(1)}{\sqrt{\ln n}}
  \xrightarrow[n\to\infty]{} -\infty.
$$

Отсюда $|c\_{q\_n}|^{1/q\_n}\to0$ и тем более $\displaystyle\limsup\_{k\to\infty}|c\_k|^{1/k}=0$.

\paragraph{3.  Формула Коши--Адамара.} Радиус сходимости равен

$$
  R=\frac{1}{\displaystyle\limsup_{k\to\infty}|c_k|^{1/k}}=+\infty.
$$

\subsection\*{Ответ}
$R=\infty$.

%%%%%%%%%%%%%%%%%%%%%%%%%%%%%%%%%%%%%%%%%%%%%%%%%%%%%%%%%%%%%%%%%%%%%%%%%%%%%%
\section{Суперэллипс и правильный декагон}

\subsection\*{Формулировка}
Найти наибольшее $a>0$, при котором кривая

$$
  \frac{x^{8}}{4}+\frac{y^{8}}{239}=a
$$

может быть целиком размещена (при некотором движении) внутри правильного десятиугольника со стороной $1$.

\subsection\*{Решение}

\paragraph{1.  Масштабирование.} Пусть $S\_a$ обозначает данную кривую. Замена $(x,y)\mapsto a^{1/8}(x,y)$ переводит $S\_1$ в $S\_a$. Поэтому достаточно исследовать $S\_1$ и найти максимальный коэффициент подобия $k\_{\max}$, с которым $S\_1$ умещается в декагон. Тогда $a\_{\max}=k\_{\max}^{8}$.

\paragraph{2.  Полярные радиус--векторы.} В полярных координатах

$$
  \rho_{S_1}(\theta)=\Bigl(\frac{\cos^{8}\theta}{4}+\frac{\sin^{8}\theta}{239}\Bigr)^{-1/8},\qquad
  \rho_{\text{dec}}(\theta)=\frac{r_{\text{in}}}{\cos\delta(\theta)},\ \ r_{\text{in}}=\frac{1}{2\tan18^{\circ}}.
$$

Здесь $\delta(\theta)$ — угол между направлением $\theta$ и ближайшей нормалью к стороне десятиугольника.

\paragraph{3.  Функция отношения.} Определим

$$
  f(\theta)=\frac{\rho_{\text{dec}}(\theta)}{\rho_{S_1}(\theta)}.
$$

Минимум этой $36^{\circ}$‑периодической функции на участке $0<\theta<18^{\circ}$ находится из условия $f'(\theta)=0$, что даёт единственный корень $\theta\_\ast\approx62.01^{\circ}$ (в глобальных координатах) и соответствующий $k\_{\max}=f(\theta\_\ast)\approx0.720489364009$.

\paragraph{4.  Переход к $a\_{\max}$.} Отсюда

$$
  a_{\max}=k_{\max}^{8}\approx0.07261403873.
$$

\subsection\*{Ответ}
$a\_{\max}\approx0.07261403873$.

%%%%%%%%%%%%%%%%%%%%%%%%%%%%%%%%%%%%%%%%%%%%%%%%%%%%%%%%%%%%%%%%%%%%%%%%%%%%%%
\section{Число неотрицательных собственных значений (размер $2025$)}

\subsection\*{Формулировка}
Дана симметричная матрица $A\in M\_{2025}(\mathbb R)$ со следующими ненулевыми диагоналями:
\begin{itemize}
\item главная диагональ: $2$;
\item диагонали сдвига $\pm14$: $-1$;
\item в первой строке и первом столбце на позициях $2,\dots,14$ стоит $1$.
\end{itemize}
Найти число неотрицательных собственных значений.

\subsection\*{Решение}
\paragraph{1.  Разложение $A=B+P$.} Пусть $B$ содержит лишь три упомянутые ленточные диагонали. Тогда $P=A-B$ имеет ненулевые элементы только в первой строке и первом столбце, следовательно $\operatorname{rk}P\le2$ (на самом деле точный ранг\~2).

\paragraph{2.  Положительная определённость $B$.} Переставив строки и столбцы по остаткам $\bmod 14$, разбиваем $B$ на $14$ независимых трёхдиагональных блоков типа

$$
  T_L=\begin{pmatrix}
    2 & -1 &       &        \\
    -1& 2 & \ddots &        \\
       &\ddots&\ddots&-1\\
       &       &-1&2
  \end{pmatrix}_{L\times L},\qquad
  L\in\{144,145\}.
$$

У такой матрицы все собственные значения равны $2-2\cos(\pi k/(L+1))>0$. Значит у $B$ все $2025$ собственных чисел положительны.

\paragraph{3.  Влияние рангового возмущения.} Так как $\operatorname{rk}P\le2$, знаки могут измениться не более чем у двух собственных чисел. Рассмотрев действие $A$ на плоскости $\operatorname{span}{e\_1,\sum\_{j=2}^{14}e\_j}$, убеждаемся, что ровно одно собственное значение становится отрицательным, второе — остаётся положительным. Остальные $2023$ наследуют знак $+$ от $B$.

\subsection\*{Ответ}
$2024$ неотрицательных собственных значений.

%%%%%%%%%%%%%%%%%%%%%%%%%%%%%%%%%%%%%%%%%%%%%%%%%%%%%%%%%%%%%%%%%%%%%%%%%%%%%%
\section{Восстановление диагонали 4,$\times$,4‑матрицы}

\subsection\*{Формулировка}
Частично известна матрица

$$
  A=\begin{pmatrix}
    * & 1 & -2 & 4\\
    1 & * & 3 & 1\\
    -2 & 3 & * & 18\\
    -2 & 3 & -2 & *
  \end{pmatrix},
\qquad
\operatorname{rk}A=2,\;\operatorname{tr}A=14.5.
$$

Найти скрытые диагональные элементы и округлить каждый до ближайшего \emph{большего} целого.

\subsection\*{Решение}
\paragraph{1.  Вывод $d$.} У третьей и четвёртой строк первые три координаты совпадают, значит их разность $(0,0,0,18-d)$ лежит в ядре при ранге $2$, откуда $d=18$.

\paragraph{2.  Вывод $a,b$.} Требуем, чтобы третья строка выражалась через две первые: $(-2,3,-2,18)=p(a,1,-2,4)+q(1,b,3,1)$. Решая систему, получаем $p=4,q=2$ и отсюда $a=-1$, $b=-\tfrac12$.

\paragraph{3.  Вывод $c$.} Из следа: $a+b+c+d=14.5\Rightarrow c=-2$.

\paragraph{4.  Округление вверх.} $(-1,-0.5,-2,18)\mapsto(-1,0,-2,18)$.

\subsection\*{Ответ}
$-1;;0;;-2;;18$.

%%%%%%%%%%%%%%%%%%%%%%%%%%%%%%%%%%%%%%%%%%%%%%%%%%%%%%%%%%%%%%%%%%%%%%%%%%%%%%
\section{Минимальный многочлен и его значение в точке $25$}

\subsection\*{Формулировка}
Оператор $\varphi$ на $\mathbb R^3$ переводит $u\mapsto v$, $v\mapsto w$, $w\mapsto u$ (то есть осуществляет циклический сдвиг по трём независимым векторам). Найти $\mu\_{\varphi}(25)$, где $\mu\_{\varphi}$ — приведённый минимальный многочлен.

\subsection\*{Решение}
Матрица $\varphi$ в базисе $(u,v,w)$ равна

$$
  P=\begin{pmatrix}0&0&1\\1&0&0\\0&1&0\end{pmatrix},
$$

со спектром ${1,\omega,\overline\omega}$, где $\omega^3=1$, $\omega\ne1$. Поэтому

$$
  \mu_{\varphi}(x)=x^3-1,\qquad \mu_{\varphi}(25)=25^3-1=15624.
$$

\subsection\*{Ответ}
$15624.000000000$.

%%%%%%%%%%%%%%%%%%%%%%%%%%%%%%%%%%%%%%%%%%%%%%%%%%%%%%%%%%%%%%%%%%%%%%%%%%%%%%
\section{Слова длины 9, не более двух символов и циклические сдвиги}

\subsection\*{Формулировка}
Алфавит из $25$ символов, $n=9$. Слово называется допустимым, если в нём использовано не более двух различных символов. Две строки эквивалентны, если одна получается из другой циклическим сдвигом. Найти максимальное возможное количество попарно неэквивалентных допустимых слов.

\subsection\*{Решение}
По лемме Бернсайда количество орбит действия группы вращений $C\_9$ равно

$$
  \frac1{9}\sum_{k=0}^{8}|\operatorname{Fix}(k)|,
$$

где $|\operatorname{Fix}(k)|$ — число слов, неизменных при сдвиге на $k$ позиций. Пусть $d:=\gcd(9,k)\in{1,3,9}$. Сдвиг разбивает позиции на $d$ циклов, каждую из которых нужно покрасить одинаковыми буквами.

\paragraph{Число раскрасок.} Используется не более двух букв.
\begin{itemize}
\item Один цвет: $25$ вариантов.
\item Два разных: $\binom{25}{2}=300$ способов выбрать буквы и $2^{d}-2$ — распределить их по циклам.
\end{itemize}
Итого

$$
  N(d)=25+300(2^{d}-2).
$$

Подставляя $d=1,3,9$ и умножая на кратности (соответственно 6, 2, 1), получаем

$$
  |V_{\max}|=\frac{153025+2\cdot1825+6\cdot25}{9}=17425.
$$

\subsection\*{Ответ}
$17425$.

%%%%%%%%%%%%%%%%%%%%%%%%%%%%%%%%%%%%%%%%%%%%%%%%%%%%%%%%%%%%%%%%%%%%%%%%%%%%%%



\end{document}
