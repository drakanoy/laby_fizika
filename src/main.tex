\documentclass{article}
\usepackage[utf8]{inputenc}
\usepackage[russian]{babel}
\usepackage{amsmath}
\usepackage{amssymb}
\usepackage{textcomp}

\begin{document}

\section*{Решение задачи (в стиле пошаговой инструкции)}

Ниже разбираем задачу достаточно подробно и <<шаг за шагом>>, чтобы был понятен общий метод решения рассеяния эллиптически поляризованного света на таком <<составном параболоиде>>. Стиль ответа немного неформальный, как просили.

\subsection*{1. Геометрия задачи и исходные данные}

\begin{enumerate}
  \item \textbf{Фигура.} Дана фигура, образованная склеиванием двух параболоидов вращения с радиусом $R$ и высотой $h = \tfrac{3}{2} R$. Ось фигуры совмещена с осью $x$. Центр – в начале координат. Уравнение (в упрощённом виде):
  \[
    |x| < h\Bigl(1 - \tfrac{y^2 + z^2}{R^2}\Bigr).
  \]
  Диэлектрическая проницаемость: $\varepsilon = 9$.

  \item \textbf{Падающая волна.} Распространяется вдоль оси $z$, имеет эллиптическую поляризацию с параметрами Стокса:
  \[
    \xi_1 = 0,\quad \xi_2 = \tfrac{24}{25},\quad \xi_3 = -\tfrac{7}{25}.
  \]
  Эти $\xi_i$ обычно определяются как $\xi_i = S_i / S_0$, где $S_i$ – компоненты вектора Стокса падающей волны.

  \item \textbf{Что нужно найти.} Требуется определить параметры Стокса рассеянной волны $\bigl(\xi_1', \, \xi_2', \, \xi_3'\bigr)$ в направлении, заданном сферическими координатами $\theta$ и $\varphi$, причём $\theta$ – угол с осью $z$, а $\varphi$ – угол между плоскостью рассеяния и плоскостью $xz$.

  \item \textbf{Подсказка.} В условии советуют вычислить <<коэффициенты деполяризации>> и <<поляризуемости>> $\alpha_i$ вдоль главных осей (здесь $x,y,z$):
  \[
    \alpha_i = \frac{p_i}{E_{0,i}}, \quad i = x,y,z,
  \]
  где $p_i$ – проекции индуцированного дипольного момента, а $E_{0,i}$ – проекции падающего электрического поля.
\end{enumerate}

\subsection*{2. Выбор аппроксимации и идея решения}

Чаще всего подобные задачи (особенно если размер фигуры мал по сравнению с длиной волны) решают в \textbf{дипольном (или квазистатическом) приближении}:

\begin{enumerate}
    \item \textbf{Предположение о малости.} Считаем, что фигура намного меньше длины волны $\lambda$. Тогда рассеяние описывается как рассеяние на электрическом диполе, который индуцируется внешним полем внутри частицы.

    \item \textbf{Главные оси и поляризуемости.} В квазистатической модели поляризуемость вдоль главных осей эллипсоидальной (или близкой к эллипсоидальной) частицы задаётся формулой:
    \[
      \alpha_i = \frac{V\,(\varepsilon - 1)}{1 + (\varepsilon - 1)\,N_i},
    \]
    где $V$ – объём частицы, а $N_i$ – деполяризационный коэффициент вдоль $i$-й оси. Для сферических или эллипсоидальных тел $N_x,N_y,N_z$ можно взять из стандартных таблиц.

    В твоём случае тело не строго эллипсоид, а \emph{составной параболоид}. Тем не менее при не слишком большом вытягивании его часто \emph{приближённо} заменяют <<prolate spheroid>> (вытянутым эллипсоидом), у которого две полуоси равны $R$ (по $y$ и $z$), а третья равна $h$ (по $x$). Тогда можно применять формулы для prolate spheroid, считая $a = R$, $c = h$.

    \item \textbf{Итоговое дипольное поле.} Индуцированный диполь $\mathbf{p} = \boldsymbol{\alpha}\,\mathbf{E}_0$. Далее, в дальней зоне (far field) амплитуда рассеянного поля $\mathbf{E}_\text{scat}$ пропорциональна $\mathbf{p}\times(\mathbf{\hat{n}}\times\mathbf{p})$ и зависит от угла $\theta, \varphi$.

    \item \textbf{Вычисление параметров Стокса.} Имея $\mathbf{E}_\text{scat}$ во вращающемся базисе (или в базисе $\mathbf{e}_\theta, \mathbf{e}_\varphi$) для данного направления $\theta, \varphi$, можно выписать компоненты (скажем, $E_{\theta}, E_{\varphi}$). Затем стандартными формулами переходят к $\mathbf{S}' = (S_0', S_1', S_2', S_3')$. И уже нормируют их на $S_0'$, получая $\xi_i' = S_i'/S_0'$.
\end{enumerate}

\subsection*{3. Формальная последовательность шагов}

\textbf{Шаг 1. Определи деполяризационные факторы $N_i$.}

Для вытянутого сфероида с полуосями $(c,a,a)$ (пусть $c = h$ вдоль $x$, а $a = R$ вдоль $y,z$) известны аналитические выражения. Если $c > a$ (prolate), то
\[
  N_x \;=\; \frac{1-e^2}{e^3}\Bigl(\tfrac12 \ln\!\frac{1+e}{1-e} \;-\; e\Bigr),
  \quad
  N_y = N_z = \tfrac{1}{2}\bigl(1 - N_x\bigr),
\]
где $e^2 = 1 - \tfrac{a^2}{c^2}$.

\textbf{Шаг 2. Поляризуемости $\alpha_i$.}

Пусть объём приблизительно $V \approx \tfrac43\pi a^2 c$. Тогда:
\[
  \alpha_x = \frac{V(\varepsilon - 1)}{1 + (\varepsilon - 1)\,N_x},\quad
  \alpha_y = \alpha_z = \frac{V(\varepsilon - 1)}{1 + (\varepsilon - 1)\,N_y}.
\]
Зная $\alpha_i$, можно найти индуцированный диполь $\mathbf{p} = (p_x, p_y, p_z)$, где
\[
  p_i = \alpha_i\,E_{0,i}.
\]

\textbf{Шаг 3. Падающая волна и её разложение по осям.}

У нас эллиптическая поляризация со Стоксами $\xi_1=0$, $\xi_2=24/25$, $\xi_3=-7/25$.
\begin{itemize}
  \item $\xi_1=0$ говорит о равенстве интенсивностей <<по горизонтали и вертикали>> (если смотреть на волновой вектор вдоль $z$).
  \item $\xi_2\neq0$ и $\xi_3\neq0$ отражают фазовые сдвиги и форму эллипса.
\end{itemize}
Чтобы найти $E_{0,x}, E_{0,y}, E_{0,z}$, надо аккуратно учесть, как волна падает вдоль $z$. Обычно для волны, идущей вдоль $z$, компоненту $E_{0,z}$ часто берут близкой к нулю в приближении плоской поперечной волны, а основные компоненты – $E_{0,x}$ и $E_{0,y}$. Но форму эллиптической поляризации можно задать так:
\[
  \mathbf{E}_0 =
  \begin{pmatrix}
  E_{0x}\\[6pt]
  E_{0y}\\[6pt]
  0
  \end{pmatrix}
  \quad \text{с некоторым относительным фазовым сдвигом.}
\]
Далее, используя связь с $\xi_2,\xi_3$, ты можешь точно восстановить фазу и амплитуды.

\textbf{Шаг 4. Индуцированный диполь $\mathbf{p}$.}

Так как $\mathbf{p} = (\alpha_x E_{0,x}, \,\alpha_y E_{0,y},\, \alpha_z E_{0,z})$, в нашем случае $E_{0,z}\approx 0$, значит $p_z \approx 0$. Главные компоненты будут $p_x$ и $p_y$.

\textbf{Шаг 5. Дальняя зона и поле рассеяния.}

В дипольном приближении поле в направлении $\mathbf{\hat{n}}$ (с углами $\theta,\varphi$) пропорционально
\[
  \mathbf{E}_\text{scat} \;\propto\;
  \mathbf{\hat{n}} \times (\mathbf{p} \times \mathbf{\hat{n}}).
\]
В сферических координатах базис можно брать $(\mathbf{e}_r,\mathbf{e}_\theta,\mathbf{e}_\varphi)$. Но чаще для поляризации используют пару $(\mathbf{e}_\theta,\mathbf{e}_\varphi)$.

\textbf{Шаг 6. Параметры Стокса рассеянной волны.}

\begin{enumerate}
  \item Находишь компоненты $E_{\theta}, E_{\varphi}$ рассеянной волны.
  \item Вычисляешь вектор Стокса по стандартным формулам:
  \[
    S_0' \;=\; |E_{\theta}|^2 + |E_{\varphi}|^2,
    \qquad
    S_1' \;=\; |E_{\theta}|^2 - |E_{\varphi}|^2,
  \]
  \[
    S_2' \;=\; 2\,\Re\{E_{\theta}E_{\varphi}^*\},
    \quad
    S_3' \;=\; 2\,\Im\{E_{\theta}E_{\varphi}^*\}.
  \]
  \item Наконец, $\xi_i' = S_i'/S_0'$. Это и будут искомые параметры Стокса в направлении $(\theta,\varphi)$.
\end{enumerate}

\subsection*{4. Короткая иллюстрация на простом примере}

Для закрепления возьмём \emph{линейно поляризованную} волну (по $x$) и \emph{сферу} (то есть $\alpha_x=\alpha_y=\alpha_z$). Тогда:

\begin{itemize}
  \item Индуцированный диполь $\mathbf{p} = \alpha (E_{0}, 0, 0)$.
  \item Поле рассеяния $\mathbf{E}_\text{scat} \propto \mathbf{\hat{n}}\times(\mathbf{p}\times\mathbf{\hat{n}})$. Если $\mathbf{\hat{n}}$ лежит под углом $\theta$ к оси $z$, то подставляем $\mathbf{p}=(p_x,0,0)$ и $\mathbf{\hat{n}}=(\sin\theta\cos\varphi,\sin\theta\sin\varphi,\cos\theta)$. Получаем компоненты, из которых нетрудно найти $S_0', S_1'$, и т.д.
\end{itemize}

У нас случай сложнее (эллиптическая поляризация + неравные $\alpha_i$), но суть та же.

\subsection*{5. Итоги}

\begin{enumerate}
  \item \textbf{Сначала} (по подсказке) ищешь $\alpha_x, \alpha_y, \alpha_z$ через деполяризационные коэффициенты.
  \item \textbf{Далее} раскладываешь падающее поле по $x,y,z$. С учётом эллиптической поляризации тебе придётся аккуратно восстановить реальные и мнимые части амплитуд $E_{0,x}, E_{0,y}$ так, чтобы удовлетворить $\xi_2, \xi_3$.
  \item \textbf{Находишь} $\mathbf{p} = \boldsymbol{\alpha}\,\mathbf{E}_0$.
  \item \textbf{Вычисляешь} рассеянное поле $\mathbf{E}_\text{scat}(\theta,\varphi)$ в дипольном приближении.
  \item \textbf{Считаешь} $\xi_1', \xi_2', \xi_3'$ по стандартным формулам для параметров Стокса.
\end{enumerate}

Таким образом, задача сводится к технике расчёта индуцированного диполя и переводу его в наблюдаемое поле в дальней зоне. Точная подстановка чисел (например, $\varepsilon=9$, $h=1.5\,R$, $\xi_2=24/25$, и т.д.) уже дело <<чистой вычислительной рутины>>, где аккуратно прогонишь формулы.

Если нужен финальный вид $\xi_i'(\theta,\varphi)$, то обычно получают выражения вида
\[
  \xi_i'(\theta,\varphi) \;=\;
  \frac{\text{(комбинации } \alpha_x,\alpha_y,\alpha_z,\; E_{0,x},E_{0,y}, \theta,\varphi)}{\text{(другая комбинация тех же величин)}},
\]
причём сами <<комбинации>> могут быть громоздкими, поэтому их часто в явном виде не пишут, ограничиваясь \emph{алгоритмом} расчёта.

\subsubsection*{Заключение}

В итоге \emph{решение} в общем виде выглядит как описанный \textbf{пошаговый алгоритм}:
\begin{enumerate}
  \item Принимаешь модель дипольного рассеяния.
  \item Используешь приближённую форму поляризуемости вдоль главных осей.
  \item Раскладываешь поле, находишь $\mathbf{p}$.
  \item Находишь $\mathbf{E}_\text{scat}$.
  \item Считаешь $\xi_1', \xi_2', \xi_3'$ по стандартным формулам для параметров Стокса.
\end{enumerate}

Подстановка конкретных чисел – это уже <<техническая>> часть, зато основные формулы даны выше. Если аккуратно всё <<проткнуть>> числами, то получится явная зависимость $\xi'_i(\theta,\varphi)$.

\end{document}